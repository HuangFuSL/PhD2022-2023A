\documentclass[aspectratio=169]{beamer}

\usepackage[UTF8]{ctex}
\usepackage{amsmath}
\usepackage{setspace}

\setstretch{1.5}

\title{Productivity, Business Profitability and Consumer Surplus: Three Different Measures of Information Technology Value}
\author{皇甫硕龙}

\begin{document}
    \maketitle

    \begin{frame}
        \frametitle{Introduction}
    
        背景问题:信息技术创造了多少商业价值?

        相关文献:时间区间?产业门类?商业价值如何定义和衡量?
    
    \end{frame}

    \begin{frame}
        \frametitle{Problem}
    
        企业要不要花费一系列成本进行信息化改造?

        \begin{enumerate}
            \item IT对企业的盈利能力有何影响?
            \item IT是否有利于提高企业的市场竞争力?
            \item IT是否增加了消费者剩余,为消费者创造了更多价值?
        \end{enumerate}
    
    \end{frame}

    \begin{frame}
        \frametitle{Knowledge}

        市场均衡理论、信息技术资产的估算方式

        \begin{equation}
            \text{IT Stock} = \text{Computer Capital} + 3\times \text{IS Labor}
        \end{equation}
    
        \begin{enumerate}
            \item 自由竞争与包含进入退出壁垒的市场竞争理论
            \item 生产函数与边际收益理论,即在自由进入退出的市场环境下,产品的销售价格等于企业生产一个单位的边际成本。
            \item 消费者剩余理论,使用消费者剩余衡量IT为消费者创造的价值
        \end{enumerate}
    
    \end{frame}

    \begin{frame}
        \frametitle{Instrument}
    
        \begin{itemize}
            \item 定性研究:微观经济学模型;
            \item 定量研究:多元线性回归(OLS、ISUR)、假设检验等。
        \end{itemize}
    
    \end{frame}
\end{document}