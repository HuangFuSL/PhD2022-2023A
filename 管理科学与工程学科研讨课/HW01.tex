\documentclass{article}
\usepackage[UTF8]{ctex}

\title{管理科学与工程学科研讨课 HW01}
\author{皇甫硕龙}

\begin{document}
    \maketitle

    \subsection*{The productivity paradox of information technology}

    \paragraph*{Knowledge} 会计学理论与微观经济学理论
    \paragraph*{Problem} 现实问题:随着信息技术的发展,计算机的算力成本按照指数级下降,但企业尤其是服务型企业的生产力没有明显的提高。文章实际讨论的问题是为什么信息技术在表观上没有增加企业生产力的相关原因。
    \paragraph*{Instrument} 文献调研

    \subsection*{Productivity, Business Profitability and Consumer Surplus}

    \paragraph*{Knowledge} 生产函数与边际收益理论、自由竞争与包含进入退出壁垒的市场竞争理论、消费者剩余理论。

    \paragraph*{Problem} 本文研究的主要问题是企业的信息化是否能影响了企业创造的商业价值?反映在现实中,企业要不要花费一系列成本进行信息化改造?根据微观经济学理论,价值可以被创造,也可以在不同的市场参与者之间转移。在本模型中,作者分别讨论了IT对企业盈利能力、企业在市场上的竞争力与消费者价值的影响,因此,文章对该问题的讨论实际上分为三个方面展开,即:

    \begin{enumerate}
        \item IT对企业的盈利能力有何影响?
        \item IT是否有利于提高企业的市场竞争力?
        \item IT是否增加了消费者剩余,为消费者创造了更多价值?
    \end{enumerate}

    \paragraph*{Instrument} 微观经济学模型、假设检验、多元线性回归(OLS、ISUR)。

    \subsection*{Information Technology Effects on Firm Performance as Measured by Tobin's $q$}

    \paragraph*{Knowledge} 用于衡量企业运营表现的Tobin's $q$指标,可以用于解释企业经营表现和一系列外部因素的关系。
    \paragraph*{Problem} 随着信息技术的发展,企业更多的借助信息化增加企业生产力以提高企业运营表现。但传统的基于会计报表的分析方法忽略了信息技术对企业经营表现的贡献,同时难以计量无形的信息资产。因此本文章主要研究的问题是,如何衡量信息技术为企业创造的商业价值。
    \paragraph*{Instrument} Tobin's $q$、多元线性回归、假设检验。
\end{document}