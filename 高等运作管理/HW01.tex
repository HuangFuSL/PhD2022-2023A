\documentclass{article}
\usepackage[UTF8]{ctex}
\usepackage{amsmath}
\usepackage{amssymb}
\usepackage{booktabs}

\title{高等运作管理 HW01}
\author{皇甫硕龙}

\newcommand{\N}{\mathbb N}
\newcommand{\R}{\mathbb R}
\newcommand{\Z}{\mathbb Z}
\newcommand{\C}{\mathbb C}
\newcommand{\dd}{\mathrm d}
\renewcommand{\geq}{\geqslant}
\renewcommand{\leq}{\leqslant}
\newcommand{\eqnref}[1]{(\ref{#1})}
\newcommand{\prob}[1]{P\left(#1\right)}
\newcommand{\cprob}[2]{P\left(#1|#2\right)}
\newcommand{\bs}{\boldsymbol}

\begin{document}
    \maketitle

    \paragraph*{a} 设利润为$P$,定义第$1\leq i\leq 6$个月内与生产相关的如下变量(其中\textbf{加粗}部分为优化变量)

    \begin{enumerate}
        \item 单位原材料成本$c_m$、单位外包成本$c_e$、单位库存成本$c^+_s$、单位缺货成本$c^-_s$、单位雇佣成本$c^+_h$、单位解雇成本$c^-_h$、单位工时成本$c_w$、单位加班成本$c'_w$;
        \item 销售单价$u$、单位产品的生产工时$t$、员工正常工时$\omega$、员工最大加班工时$\omega'$;
        \item 经济变量:收益$R_i$,总成本$C_i$,净利润$P_i$;
        \item 供求变量:需求$D_i$,月末库存量$T^+_i$,月末缺货量$T^-_i$;
        \item 生产变量:产量$L_i$,\textbf{生产工时}$W_i$,\textbf{加班工时}$W'_i$、\textbf{外包量}$E_i$;
        \item 雇佣变量:员工数量$w_i$,\textbf{月初雇佣员工数量}$H^+_i$,\textbf{月初解雇员工数量}$H^-_i$;
    \end{enumerate}

    目标函数为
    
    \begin{align}
        & \text{总利润} & \max P &= \sum_{i=1}^6 P_i
    \end{align}
    
    变量间服从如下关系:

    \begin{align}
        & \text{初始条件} & T^+_i = 1000, T^-_i &= 0, w_i = 80 \\
        & \text{净利润} & P_i &= R_i - C_i \\
        & \text{销售收入} & R_i &= uD_i \\
        & \text{销售与库存} & D_i + T^+_i - T^-_i &= T^+_{i-1} - T^-_{i-1} + L_i + E_i \\
        & \text{生产工时} & L_i &= (W_i + W'_i)/t \\
        & \text{员工数量} & w_i &= w_{i-1} + H^+_i - H^-_i \\
        & \text{总成本} & C_i &= c_mL_i + c_eE_i + c^+_sT^+_i + c^-_sT^-_i \\
        & & & + c^+_hH^+_i + c^-_hH^-_i + \omega c_ww_i + c'_wW'_i
    \end{align}

    变量满足如下约束条件:

    \begin{align}
        & \text{最终条件} & T^+_6 &\geq 500 \\
        & \text{不允许缺货} & T^-_0&= 0 \\
        & \text{加班工时约束} & W'_i &\leq 10w_i \\
        & \text{正常工时约束} & W_i &\leq 160w_i \\
        & \text{整数约束} & H^+_i &\in \N, H^-_i \in \N \\
    \end{align}

    使用Excel求解线性规划,得$P=217340$,参数如表\ref{tbl:final-result}所示:

    \begin{table}[hp]
        \centering
        \caption{线性规划最优解}
        \begin{tabular}{c|cccccc}
            \toprule
            参数 & 1 & 2 & 3 & 4 & 5 & 6 \\
            \midrule
            净利润$P_i$ & $-14480$ & $50400$ & $59680$ & $80140$ & $21160$ & $20440$ \\
            收益$R_i$ & $64000$ & $120000$ & $128000$ & $152000$ & $88000$ & $88000$ \\
            总成本$C_i$ & $78480$ & $69600$ & $68320$ & $71860$ & $66840$ & $67560$ \\
            需求$D_i$ & $1600$ & $3000$ & $3200$ & $3800$ & $2200$ & $2200$ \\
            月末库存$T^+_i$ & $1960$ & $1520$ & $880$ & $0$ & $140$ & $500$ \\
            月末缺货$T^-_i$ & $0$ & $0$ & $0$ & $220$ & $0$ & $0$ \\
            产量$L_i$ & $2560$ & $2560$ & $2560$ & $2560$ & $2560$ & $2560$ \\
            生产工时$W_i$ & $10240$ & $10240$ & $10240$ & $10240$ & $10240$ & $10240$ \\
            加班工时$W_i'$ & $0$ & $0$ & $0$ & $0$ & $0$ & $0$ \\
            外包量$E_i$ & $0$ & $0$ & $0$ & $140$ & $0$ & $0$ \\
            员工数量$w_i$ & $64$ & $64$ & $64$ & $64$ & $64$ & $64$ \\
            月初雇佣数量$H^+_i$ & $0$ & $0$ & $0$ & $0$ & $0$ & $0$ \\
            月末解雇数量$H^-_i$ & $16$ & $0$ & $0$ & $0$ & $0$ & $0$ \\
            \bottomrule
        \end{tabular}
        \label{tbl:final-result}
    \end{table}

    \paragraph*{b} 1月促销时最优利润为$P' = 211320$,4月促销时最优利润为$P'' = 211220$。因此1月促销更为有效。

    \paragraph*{c} 1月促销时最优利润为$P' = 243320$,4月促销时最优利润为$P'' = 247320$。因此4月促销更为有效。原因:4月需求本身比1月高,但5、6月的需求比2、3月的需求低,因此在促销对增加当月需求的影响较小时,促销所造成的需求转移因素占主要地位,使得1月比4月更优;而当促销对增加当月需求的影响较大时,促销所造成的需求增加因素占主要地位,使得4月比1月更优。

    \paragraph*{d} 在旺季来临前的淡季开始促销,可能使销售旺季提前开始。

\end{document}