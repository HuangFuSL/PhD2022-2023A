\documentclass{../notes}

\title{随机需求下的库存管理}

\begin{document}
    \maketitle

    研究需求随机时的库存管理。

    随机库存控制模型:

    \begin{enumerate}
        \item 周期检查策略:单周期、有限周期、无限周期(影响因素:固定订货成本);事件顺序、基于期末库存计算成本
        \item 连续检查策略
        \item 计算最优策略与最优参数
    \end{enumerate}

    常用概率分布:泊松分布(顾客到达)、正态分布(消费量)

    \section*{报童模型}

    特点:只能订一次货、产品易变质、新旧产品不可混合

    参数:\begin{enumerate}
        \item 需求$D$,$D$服从概率密度函数$f$,分布函数$F$
        \item 订货量$Q$
        \item 缺货成本$c_u$、库存成本$c_o$
    \end{enumerate}

    将成本表示为随机变量$D$和决策变量$Q$的函数,通过确定$Q$使得函数的期望最小。即$C(Q, D) = c_0\max\{0, Q-D\} + c_u\max\{0, D-Q\}$,计算期望成本函数$C(Q) = \cexpt{C(Q, D)}Q$

    \begin{align}
        \text{期望缺货量}\quad & n(Q) = \expt{(X-Q)^+} = \int_Q^\infty (x-Q)f(x)\dd x \\
        \text{期望持有库存}\quad & \bar n(Q) = \expt{(Q-X)^+} = \int_{0}^Q (Q-x)f(x)\dd x \\
        \text{期望成本}\quad & C(Q) = c_u n(Q) + c_o\bar n(Q)
    \end{align}

    结论:$C'(Q) = c_oF(Q) + c_u(1-F(Q))$,因此最优订货量满足

    \begin{equation}
        F(Q) = \prob{D\leq Q} = \frac{c_u}{c_o+c_u}
    \end{equation}

    \begin{itemize}
        \item 当$c_u$增加时,缺货导致的损失增加,零售商倾向于提高订货量以避免缺货损失
        \item 计算正态分布需求下的最优成本?
    \end{itemize}
\end{document}