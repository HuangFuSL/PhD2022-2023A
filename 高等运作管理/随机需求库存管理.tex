\documentclass{../notes}

\title{随机需求下的库存管理}

\begin{document}
    \maketitle

    研究需求随机时的库存管理。

    随机库存控制模型:

    \begin{enumerate}
        \item 周期检查策略:单周期、有限周期、无限周期(影响因素:固定订货成本);事件顺序、基于期末库存计算成本
        \item 连续检查策略
        \item 计算最优策略与最优参数
    \end{enumerate}

    常用概率分布:泊松分布(顾客到达)、正态分布(消费量)

    \section*{报童模型}

    \subsection*{单周期}

    特点:只能订一次货、产品易变质、新旧产品不可混合

    参数:\begin{enumerate}
        \item 需求$D$,$D$服从概率密度函数$f$,分布函数$F$
        \item 订货量$Q$,进货价格$c$,销售价格$s$
        \item 缺货成本$c_u$、库存成本$c_o$
    \end{enumerate}

    将成本表示为随机变量$D$和决策变量$Q$的函数,通过确定$Q$使得函数的期望最小。即$C(Q, D) = c_0\max\{0, Q-D\} + c_u\max\{0, D-Q\}$,计算期望成本函数$C(Q) = \cexpt{C(Q, D)}Q$

    \begin{align}
        \text{期望缺货量}\quad & n(Q) = \expt{(X-Q)^+} = \int_Q^\infty (x-Q)f(x)\dd x \\
        \text{期望持有库存}\quad & \bar n(Q) = \expt{(Q-X)^+} = \int_{0}^Q (Q-x)f(x)\dd x \\
        \text{期望成本}\quad & C(Q) = c_u n(Q) + c_o\bar n(Q)
    \end{align}

    结论:$C'(Q) = c_oF(Q) + c_u(1-F(Q))$,因此最优订货量满足

    \begin{equation}
        F(Q) = \prob{D\leq Q} = \frac{c_u}{c_o+c_u}
    \end{equation}

    \begin{itemize}
        \item 当$c_u$增加时,缺货导致的损失增加,零售商倾向于提高订货量以避免缺货损失
        \item 计算正态分布需求下的最优成本?若$F$服从正态分布,可以转化为标准正态分布后使用分位数计算。
    \end{itemize}

    \separate[0.5pt]

    对于\textit{订货量$s$、成本$c$、销售价格$s$、持有成本$h$、缺货损失$p$}的销售模型,计算总损失:

    \begin{equation}
        \begin{aligned}
            g(Q) &= cQ+c_o\expt{(Q-d)^+} + c_u\expt{(d-Q)^+} - s\expt{\min(Q, d)} \\
            &= cQ + h\bar n(S) + pn(S) - s\left(\int_0^Q xf(x)\dd x + \int_Q^\infty Qf(x)\dd x\right) \\
            &= cQ - s\mu + h\bar n(S) + pn(S) + s\int_0^\infty (x-Q)f(x)\dd x
        \end{aligned}
    \end{equation}

    解方程$g'(Q) = 0$,得

    \begin{equation}
        F(Q^*) = \frac{p + s - c}{h + p + s}
    \end{equation}

    对应为$c_u = p + s - c, c_o = c + h$。

    对于正态分布,计算持有成本与缺货成本部分:

    \begin{equation}
        g'(Q) = h\bar n(S) + pn(S)
    \end{equation}

    令$y=(x-\mu) / \sigma$

    \begin{equation*}
        \begin{aligned}
            n(S) &= \int_S^\infty (x-S)f(x)\dd x \\
            n\left(\frac{S-\mu}{\sigma}\right) &= \int_{\frac{S-\mu}{\sigma}}^\infty \left(y - \frac{S-\mu}{\sigma}\right)z(y)\dd y \\
            \Rightarrow n(S)&= \sigma L\left(\frac{S-\mu}{\sigma}\right) = \sigma L(z) \\
            L(z) &= \phi(z) - z(1 - \Phi(z))
        \end{aligned}
    \end{equation*}

    对于持有成本部分

    \begin{equation*}
        \bar n(S) = \expt{(S-x)^+} = S-\mu+n(S) = \sigma \left(z_\alpha + L(z)\right)
    \end{equation*}

    最优情况下

    \begin{equation}
        \begin{aligned}
            g\left(S^*\right) &= h\sigma \left(z+ L(z)\right) + p\sigma L(z) \\
            &= \sigma\left(hz + (h + p)L(z)\right) \\
            &= \sigma(h + p)\phi(z)
        \end{aligned}
    \end{equation}

    \separate[0.5pt]

    \paragraph*{需求离散} 此时最优的$CR$位于两个离散$F(Q)$之间。该情况又可以分为两种情况:订货离散的前提下,需求是否出现离散。比较订货量分别为$Q^*, Q^*+1$的情况。有$c_u\prob {D\geq Q^*+1}\leq c_o\prob {D\leq Q^*}$,从而$F(Q^*)\geq c_u/(c_u+c_o)$

    或可使用差分方法,已知成本函数

    \begin{equation}
        g(S) = h\sum_{d=0}^S (S-d)f(d) + p\sum_{d=S}^\infty (d-S)f(d)
    \end{equation}

    定义$\Delta(S) = g(S+1) - g(S)$,则

    \begin{equation*}
        \begin{aligned}
            \Delta(S) &= g(S+1) - g(S) \\
            &=h\sum_{d=0}^{S+1} (S + 1 - d)f(d) + p\sum_{d=S+1}^\infty (d- S - 1)f(d) - h\sum_{d=0}^S (S-d)f(d) - p\sum_{d=S}^\infty (d-S)f(d) \\
            &= h\sum_{d=0}^Sf(d) + p\sum_{d=S + 1}^\infty f(d) \\
            &= hF(s) + p(1-F(s)) > 0
        \end{aligned}
    \end{equation*}

    \subsection*{周期检查模型-无固定订货成本}

    对于周期检查的报童模型,有如下参数和假设

    \begin{enumerate}
        \item 各周期需求独立同分布,概率密度函数$f(\cdot)$、概率分布函数$F(\cdot)$
        \item 贴现率$\alpha$
        \item 不考虑订货提前期
        \item 期初库存$y_0$,剩余周期数$n$,$C_n(y_0)$为当前周期到剩余$n$个周期的总期望贴现成本的最小值,即
        \begin{equation*}
            C_n\left(y_0\right) = \min_{y\geq y_0}\left\{L(y) - cy_0 + \alpha\int_0^\infty C_{n-1}[t(y, x)]f(x)\mathrm dx\right\}
        \end{equation*}
    \end{enumerate}
\end{document}