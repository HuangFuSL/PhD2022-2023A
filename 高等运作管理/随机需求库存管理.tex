\documentclass{../notes}

\title{随机需求下的库存管理}

\begin{document}
    \maketitle

    研究需求随机时的库存管理。

    随机库存控制模型:

    \begin{enumerate}
        \item 周期检查策略:单周期、有限周期、无限周期(影响因素:固定订货成本);事件顺序、基于期末库存计算成本
        \item 连续检查策略
        \item 计算最优策略与最优参数
    \end{enumerate}

    常用概率分布:泊松分布(顾客到达)、正态分布(消费量)

    \section*{离散需求:报童模型}

    \subsection*{单周期-无固定订货成本}

    特点:只能订一次货、产品易变质、新旧产品不可混合

    参数:\begin{enumerate}
        \item 需求$D$,$D$服从概率密度函数$f$,分布函数$F$
        \item 订货量$Q$,进货价格$c$,销售价格$s$
        \item 缺货成本$c_u$、库存成本$c_o$
    \end{enumerate}

    将成本表示为随机变量$D$和决策变量$Q$的函数,通过确定$Q$使得函数的期望最小。即$C(Q, D) = c_0\max\{0, Q-D\} + c_u\max\{0, D-Q\}$,计算期望成本函数$C(Q) = \cexpt{C(Q, D)}Q$

    \begin{align}
        \text{期望缺货量}\quad & n(Q) = \expt{(X-Q)^+} = \int_Q^\infty (x-Q)f(x)\dd x \\
        \text{期望持有库存}\quad & \bar n(Q) = \expt{(Q-X)^+} = \int_{0}^Q (Q-x)f(x)\dd x \\
        \text{期望成本}\quad & C(Q) = c_u n(Q) + c_o\bar n(Q)
    \end{align}

    结论:$C'(Q) = c_oF(Q) + c_u(1-F(Q))$,因此最优订货量满足

    \begin{equation}
        F(Q) = \prob{D\leq Q} = \frac{c_u}{c_o+c_u}
    \end{equation}

    \begin{itemize}
        \item 当$c_u$增加时,缺货导致的损失增加,零售商倾向于提高订货量以避免缺货损失
        \item 计算正态分布需求下的最优成本?若$F$服从正态分布,可以转化为标准正态分布后使用分位数计算。
    \end{itemize}

    \separate[0.5pt]

    对于\textit{订货量$s$、成本$c$、销售价格$s$、持有成本$h$、缺货损失$p$}的销售模型,计算总损失:

    \begin{equation}
        \begin{aligned}
            g(Q) &= cQ+c_o\expt{(Q-d)^+} + c_u\expt{(d-Q)^+} - s\expt{\min(Q, d)} \\
            &= cQ + h\bar n(S) + pn(S) - s\left(\int_0^Q xf(x)\dd x + \int_Q^\infty Qf(x)\dd x\right) \\
            &= cQ - s\mu + h\bar n(S) + pn(S) + s\int_0^\infty (x-Q)f(x)\dd x
        \end{aligned}
    \end{equation}

    解方程$g'(Q) = 0$,得

    \begin{equation}
        F(Q^*) = \frac{p + s - c}{h + p + s}
    \end{equation}

    对应为$c_u = p + s - c, c_o = c + h$。

    对于正态分布,计算持有成本与缺货成本部分:

    \begin{equation}
        g'(Q) = h\bar n(S) + pn(S)
    \end{equation}

    令$y=(x-\mu) / \sigma$

    \begin{equation*}
        \begin{aligned}
            n(S) &= \int_S^\infty (x-S)f(x)\dd x \\
            n\left(\frac{S-\mu}{\sigma}\right) &= \int_{\frac{S-\mu}{\sigma}}^\infty \left(y - \frac{S-\mu}{\sigma}\right)z(y)\dd y \\
            \Rightarrow n(S)&= \sigma L\left(\frac{S-\mu}{\sigma}\right) = \sigma L(z) \\
            L(z) &= \phi(z) - z(1 - \Phi(z))
        \end{aligned}
    \end{equation*}

    对于持有成本部分

    \begin{equation*}
        \bar n(S) = \expt{(S-x)^+} = S-\mu+n(S) = \sigma \left(z_\alpha + L(z)\right)
    \end{equation*}

    最优情况下

    \begin{equation}
        \begin{aligned}
            g\left(S^*\right) &= h\sigma \left(z+ L(z)\right) + p\sigma L(z) \\
            &= \sigma\left(hz + (h + p)L(z)\right) \\
            &= \sigma(h + p)\phi(z)
        \end{aligned}
    \end{equation}

    \separate[0.5pt]

    \paragraph*{需求离散} 此时最优的$CR$位于两个离散$F(Q)$之间。该情况又可以分为两种情况:订货离散的前提下,需求是否出现离散。比较订货量分别为$Q^*, Q^*+1$的情况。有$c_u\prob {D\geq Q^*+1}\leq c_o\prob {D\leq Q^*}$,从而$F(Q^*)\geq c_u/(c_u+c_o)$

    或可使用差分方法,已知成本函数

    \begin{equation}
        g(S) = h\bar n(S) + pn(S) = h\sum_{d=0}^S (S-d)f(d) + p\sum_{d=S}^\infty (d-S)f(d)
    \end{equation}

    定义$\Delta(S) = g(S+1) - g(S)$,则

    \begin{equation*}
        \begin{aligned}
            \Delta(S) &= g(S+1) - g(S) \\
            &=h\sum_{d=0}^{S+1} (S + 1 - d)f(d) + p\sum_{d=S+1}^\infty (d- S - 1)f(d) - h\sum_{d=0}^S (S-d)f(d) - p\sum_{d=S}^\infty (d-S)f(d) \\
            &= h\sum_{d=0}^Sf(d) + p\sum_{d=S + 1}^\infty f(d) \\
            &= hF(s) + p(1-F(s)) > 0
        \end{aligned}
    \end{equation*}

    \subsection*{有限周期检查模型-无固定订货成本}

    对于周期检查的报童模型,有如下参数和假设

    \begin{enumerate}
        \item 各周期需求独立同分布,概率密度函数$f(\cdot)$、概率分布函数$F(\cdot)$
        \item 贴现率$\alpha$
        \item 不考虑订货提前期
        \item 期初库存$y_0$,剩余周期数$n$,$C_n(y_0)$为当前周期到剩余$n$个周期的总期望贴现成本的最小值,即
        \begin{equation*}
            C_n\left(y_0\right) = \min_{y\geq y_0}\left\{L(y) - cy_0 + \alpha\int_0^\infty C_{n-1}[t(y, x)]f(x)\mathrm dx\right\}
        \end{equation*}
        \item 订货量$s$
    \end{enumerate}

    设第$t$周期时对应的库存为$s$,则$t$期的成本

    \begin{equation*}
        \theta_t(x) = \min_{y\geq x}\left\{c(y-x) + g(y) + rE_D\left[\theta_{t+1}(y-D)\right]\right\}
    \end{equation*}

    式中$\theta_{t+1}(x)$应为凸函数,存在极小值。

    当$\theta_{t+1}(x) = -cx$时,$S_t^*$相同,有

    \begin{equation*}
        \begin{aligned}
            H_T(y) &= cy + g(y) + rE_D[\theta_{t+1}(y-D)] \\
            &= (1-r)cy + g(y) + rc\mu
        \end{aligned}
    \end{equation*}

    解方程$H_T'(y) = 0$,解得

    \begin{equation*}
        S_T^* = F^{-1}\left(\frac{p - (1-r) c}{p+h}\right)
    \end{equation*}

    \begin{equation}
        Q_T(x) = \begin{cases}
            H_T(S_T^*) - cx & x\leq S_T^* \\
            H_T(x) - cx & x > S_T^*
        \end{cases}
    \end{equation}

    计算$H_{T-1}(y)$:

    \begin{derive}[H_{T-1}(y)]
        &= cy + g(y) + rE_D[\theta_T(y-D)] \\
        &= \begin{cases}
            cy + g(y) + rE_D[H_T(S_T^*) - c(y-D)] & y\leq S_T^* \\
            ? & y > S_T^*
        \end{cases}
    \end{derive}

    \subsection*{无限周期模型-无固定订货成本} 

    定义第二类服务水平

    \begin{equation*}
        B = E\left[\frac{\text{周期中被立刻满足的需求}}{\text{周期总需求}}\right]
    \end{equation*}

    \begin{enumerate}
        \item 近似1:每周期最多进一次货,$\hat B_1 = 1 - \frac{n^{L+1}(S)}{\mu}$
        \item 近似2:$E(\#/\#) \approx E(\#)/E(\#)$,$\hat B_2 = \frac{\bar n^L(S) - \bar n^{L+1}(S)}{\mu}$。$\bar n^L(S) = E\left[(S - D(L))^+\right]\approx E(S-D(L)) = SL\mu$,$\bar n^{L+1}(S) = S - (L+1)\mu + n^{L+1}(S)$
    \end{enumerate}

    当提前期$L = 0$时,有

    \begin{equation*}
        B = F(s) + \int_s ^\infty \frac{S}{d} f(d) \dd d
    \end{equation*}

    \subsection*{单周期-固定订货成本}

    固定订货成本为$K$,其他假设与报童模型相同,最优策略为$(s, Q)$策略

    % \begin{figure}[ht]
    %     \centering
    %     \begin{tikzpicture}
    %         \draw[->, domain]
    %     \end{tikzpicture}
    % \end{figure}

    证明最优策略的有效性

    \begin{derive*}[\theta (x)]
        &= \min _{y\geq x}\left\{K\times 1_{\{y > x\}} + c(y-x) + g(y)\right\} \\
        &= \min _{y\geq x}\left\{K\times 1_{\{y > x\}} + H(y) - cx\right\}
    \end{derive*}

    \subsection*{有限周期-固定订货成本}

    对于第$t$个周期,有

    \begin{derive*}[\theta (x)] 
        &= \min _{y\geq x}\left\{K\times 1_{\{y > x\}} + H_t(y) - cx\right\} \\
        &= -cx + \begin{cases}
            K + H_t(S^*) & x \leq S^* \\
            H_t(x) & x > S^* 
        \end{cases} \\
        &= -cx + \phi_t(x)
    \end{derive*}

    式中

    \begin{equation*}
        H_t(y) = cy + g(y) + \gamma E_D\left[\theta_{t+1} (y-D)\right]
    \end{equation*}

    定义:若$f(x)$满足

    \begin{equation*}
        f(x) + a\cdot \frac{f(x) - f(x-b)}{b} \leq f(x+a) + K
    \end{equation*}

    称$f(x)$为$K$凸函数。$K$凸的函数$f(x)$满足如下性质:

    \begin{enumerate}
        \item $g(x) = f(x+\varepsilon), \varepsilon\in \R$为$K$凸函数;
        \item $g(x) = \alpha_1f_1(x) + \alpha_2f_2(x), \alpha_1 \in R, \alpha_2\in R$为$K$凸函数;
        \item 设$Y$为随机变量,$E_Y\left[f(x-y)\right]$为$K$凸函数;
        \item 若$f(x)$为$K_1$凸函数,且$K_2 > K_1$,则$f(x)$也为$K_2$凸函数。
    \end{enumerate}
    
    设$S^*$为全局最低点,$s^*$为满足$f(x) = f(S^*) + K$最大的$x\leq S^*$,则最优策略为$(s, S)$。

    \begin{enumerate}
        \item $f$在$(-\infty, s^*)$上非增
        \item 若$s^* < x \leq S^*$,则$f(x) < f(s^*)$
        \item 设$S^* < x_1 < x_2$,则$f(x_1) - f(x_2) \leq K$
    \end{enumerate}

    可以通过$K$凸性证明$(s, S)$策略为最优策略。从后往前归纳证明,即:

    \begin{equation*}
        \theta_{t+1}(x)\text{为$K$凸函数}\Rightarrow \theta_t(x)\text{为$K$凸函数}
    \end{equation*}

    \begin{enumerate}
        \item $x-b > s_t^*$时,$\phi_t(y) = H_t(y), y\in (x-b, x+a)$
        \item $x+a < s_t^*$时,$\phi_t(t) = H_t(S_t^*) + K, y\in (x-b, x+a)$
        \item $x-b\leq s_t^* \leq x+a$时,进一步讨论
        \begin{enumerate}[label=\Roman*.]
            \item $\phi_t(x) \leq H_t(S_t^*) + K$
            \item $\phi_t(x) > H_t(S_t^*) + K$
        \end{enumerate} 
    \end{enumerate}
    
    \section*{连续需求}

    $(r, Q)$模型,式中$r$为重订货点,$Q$为订货量。订货后距离订货到达有提前期$L$

    \begin{equation*}
        IL = IP - D(L)
    \end{equation*}

    式中$D(L)$为$L$时间内的需求分布,$IL_{t+L}$(Inventory Level)为到达后的库存水平,$IP_{t}$(Inventory Position)为订货后的总库存水平与订货量之和。可以通过$IP$的分布推算$IL$的分布。

    $IP$的取值是非周期、不可约、有限状态、连续时间马尔可夫链。因此一定存在平稳分布,设$\bs P$为状态转移矩阵,则稳态$\bs \pi = \bs \pi \bs P$。

    考虑$IP$的分布,再考虑$P(IL|IP) = P(D(L) = IL - IP)$,通过全期望公式可以得到库存水平$IL$的期望,从而计算出库存成本的期望。

    期望成本函数$g(r, Q)$计算如下

    \begin{derive}[g(r, Q)]
        &= \frac{K\lambda + \int_{r}^{r+Q} g(y)\dd y}{Q} \\
        &= \frac{K\lambda}{Q} + \frac{h}{Q}\int_r^{r+Q} \expt{(y-D)^+} \dd y+ \frac{p}{Q}\int_r^{r+Q} \expt{(D-y)^+} \dd y \\
        &= \frac{K\lambda}{Q} + hI(r, Q) + pB(r, Q) \\
        &= \frac{K\lambda}{Q} + h\left(\frac{Q}{2}+r-\lambda L\right) + (p + h)B(r, Q)
    \end{derive}

    期望成本函数解的性质$g(r, Q) = g(r+Q) = g(r)$。可以通过迭代的方式计算数值解。
\end{document}