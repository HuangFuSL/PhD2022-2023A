\documentclass{../notes}

\title{决策理论 HW03}

\loadgeometry{word-moderate}

\begin{document}
\maketitle

\paragraph*{1} 对于第一种情况,人们更倾向于认为丢失的100元是独立于“看电影”这一事件以外的损失,即人们认为“看电影”这一事件的支出为100。而对于第二种情况,人们更倾向于认为丢失100元电影票后,花费100元重新购买电影票是“看电影”这一事件的额外损失,即人们认为“看电影”这一事件的支出为200元。由于人们感受到的支出为200元大于100元,因此第二种情形中更多的人倾向于不购买电影票。

\separate

\paragraph*{2} 当决策方案的备选项较少时,增加决策方案的数量可以为决策者增加额外的备选项,从而能够帮助决策者做出更为优化的决策。反之,当决策方案的备选项较多时,由于决策者并非完全理性,再增加备选项会通过风险类型、心理账户等因素对决策者对已有选项的收益/损失分析产生干扰,从而影响决策者作出正常决策。

\separate

\paragraph*{3} 若选择第二天结束时查看投资成果,投资结果及概率如表\ref{tbl:two-days}所示:

\begin{table}[ht]
    \centering
    \caption{第二天结束时投资成果的概率分布}
    \label{tbl:two-days}
    \begin{tabular}{*{4}{c}}
        \toprule
        投资结果$x$ & $6000$ & $2000$ & $-2000$ \\
        \midrule
        概率      & $1/16$ & $3/8$  & $9/16$  \\
        \bottomrule
    \end{tabular}
\end{table}

\subparagraph*{(1)} 当选择每天结束查看投资成果时,总的展望为

\begin{equation}
    U_1 = \left(\frac 14f(3000) + \frac 34f(-1000)\right) + \left(\frac 14f(3000) + \frac 34f(-1000)\right) = -1500
\end{equation}

当选择两天后查看投资成果时,总的展望为

\begin{equation}
    U_2 = \frac {1}{16}f(6000) + \frac 38f(2000) + \frac{9}{16}f(-2000) = -1125
\end{equation}

因此$U_2>U_1$,会选择在两天后查看投资成果。

\subparagraph*{(2)} 当第一天查看投资成果时,对于第二天,有

\begin{equation}
    U' = \frac 14f(3000) + \frac 34f(-1000) = -750 < 0
\end{equation}

因此第一天结束后会选择离开投资市场,此时总的展望为$U' = -750 < U_2$,因此会更喜欢第一种查看方式。
\end{document}