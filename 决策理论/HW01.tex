\documentclass[UTF8]{ctexart}
\usepackage{amsmath}
\usepackage{amssymb}

\title{决策理论 HW01}
\author{皇甫硕龙}

\newcommand{\N}{\mathbb N}
\newcommand{\R}{\mathbb R}
\newcommand{\Z}{\mathbb Z}
\newcommand{\C}{\mathbb C}
\newcommand{\dd}{\mathrm d}
\renewcommand{\geq}{\geqslant}
\renewcommand{\leq}{\leqslant}
\newcommand{\eqnref}[1]{(\ref{#1})}
\newcommand{\prob}[1]{P\left(#1\right)}
\newcommand{\cprob}[2]{P\left(#1|#2\right)}
\newcommand{\bs}{\boldsymbol}

\begin{document}
    \maketitle

    \paragraph*{1}

    设$P_i, i=1, 2, 3$为第$i$位投赞成票的概率,随机变量$X$为投赞成票的人数,则$X\sim B(3, p)$

    \begin{equation}
        P(Y) = P(X\geq 2) = p^3 +  3p^2(1 - p)
    \end{equation}

    \begin{equation}
        \begin{aligned}
            P(F|Y) &= \frac{P(F\cap Y)}{P(Y)} \\
            &= \frac{p(p^2 + 2p(1-p))}{p^3 +  3p^2(1 - p)} \\
            &= \frac{2-p}{3-2p}
        \end{aligned}
    \end{equation}

    因此,$\lim_{p\rightarrow 0} P(F|Y) = 2/3, \lim_{p\rightarrow 1} P(F|Y) = 1$。当$p\rightarrow 0$时,$1-p\rightarrow 1$,有$p^3\ll 3p^2(1-p)$,因此在已知通过的前提下大概率为2票通过,则每个人投赞成票的概率趋向于$2/3$。反之,当$p\rightarrow 1$时,$1-p\rightarrow 0$,有$p^3\gg 3p^2(1-p)$,因此在已知通过的前提下大概率为3票通过,则每个人投赞成票的概率趋向于$1$。

    \paragraph*{2} 降水量以一年为周期随季节波动,夏季降雨多,冬季降雨少。而每年同期的降水量可以视为独立同分布。因此,可以使用随季节变化的预测模型进行预测。设$1\leq i$,令$X_i$为一年内第$i$周的降水量,观测期为$n$年,得到$X_i$的一系列观测值$x_{ij}$。如式\eqnref{eq:1}定义第$i$周的季节因子$c_i$,则$E(X_i/c_i) = E(X_j/c_j), \forall i\not = j$。从而将季节序列转化为平稳序列,可以使用滑动窗口或指数平滑等方式进行预测。

    \begin{equation}
        c_i = \frac{\bar{X}_{i\cdot}}{\bar{X}_{\cdot\cdot}}
        \label{eq:1}
    \end{equation}
    
    该种预测方法没有考虑$X_i$并非同分布,因此$V(X_i)$通常不相等。反映在现实生活中,如中国大多数城市属温带大陆性气候,夏季高温多雨,冬季干旱少雨。夏季温度升高,形成低压区,受到太平洋季风的影响,造成降雨。因此降雨的强度取决于季风的强度,而季风的强度受到更多因素的影响,使得夏季国内同时期降水量变化幅度较大。而冬季温度降低,形成低压区,降水量普遍较少,变化幅度小。季节因子$c_i$只消除了$X_i$均值之间的差异,而没有考虑$X_i$方差之间的差异。在夏季,$X_i$变化幅度更大,更容易出现异常值(即远高于正常观测水平的降水),而季节性周期预测方法不容易预测出此类异常值。

    高于正常水平的降水量出现概率较低,通常为数十年发生一次。根据生活经验,发生严重的气候灾害前,通常会出现一系列严重程度较低的气候反常现象。因此,可以通过观测反常气候的出现周期来预测气候灾害。如使用机器学习模型或基于概率的马尔科夫链等模型进行预测。
\end{document}