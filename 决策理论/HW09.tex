\documentclass{../notes}

\title{决策理论 HW09}

\loadgeometry{word-moderate}

\begin{document}
    \maketitle

    支付矩阵列于表\ref{tbl:1}中。根据支付矩阵可知不存在纯策略Nash均衡。设Army 1以$p_i$的概率攻击地区$i$。则Army 1攻击各个地区的期望收益为

    \begin{equation}
        \begin{bmatrix}
            p_2v_2 + p_3v_3 \\
            p_1v_1 + p_3v_3 \\
            p_1v_1 + p_2v_2 \\
        \end{bmatrix}
    \end{equation}

    均衡下,Army 1供攻击各个地区的期望收益应当相等,则

    \begin{equation}
        \begin{aligned}
            p_1v_1 + p_2v_2 &= p_2v_2 + p_3v_3 \\
            p_1v_1 + p_2v_2 &= p_1v_1 + p_3v_3 \\
            p_1 + p_2 + p_3 &= 1
        \end{aligned}
    \end{equation}

    解得

    \begin{equation}
        \left\{
        \begin{aligned}
            p_1 &= \frac{v_2v_3}{v_1v_2 + v_2v_3 + v_1v_3} \\
            p_2 &= \frac{v_1v_3}{v_1v_2 + v_2v_3 + v_1v_3} \\
            p_3 &= \frac{v_1v_2}{v_1v_2 + v_2v_3 + v_1v_3}
        \end{aligned}
        \right.
    \end{equation}

    \begin{table}[h]
        \centering
        \caption{支付矩阵}
        \begin{tabular}{|c|ccc|}
            \hline
            \diagbox{Army 1}{Army 2} & 1 & 2 & 3 \\
            \hline
            1 & $0$ & $v_1$ & $v_1$ \\
            2 & $v_2$ & $0$ & $v_2$ \\
            3 & $v_3$ & $v_3$ & $0$ \\
            \hline
        \end{tabular}
        \label{tbl:1}
    \end{table}
\end{document}