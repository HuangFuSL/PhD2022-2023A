\documentclass{../notes}

\title{组合数学 HW03}

\newcommand{\Gx}[1]{G^{(#1)}(x)}

\begin{document}
    \maketitle

    \paragraph*{2.1} 设函数族$\Gx n$为形如\eqnref{eq:2.1.1}的函数

    \begin{equation}
        G^{(n)}(x) = \sum_{k=0}^\infty k^n x^k
        \label{eq:2.1.1}
    \end{equation}

    则母函数为$G(x) = \Gx 3$。错位相减,计算$G(x) - xG(x)$,得到

    \begin{derive}[G(x) - xG(x)]
        &= \sum_{n=0}^\infty n^3x^n- \sum_{n=1}^\infty n^3x^{n+1} \\
        &= \sum_{n=1}^\infty \left(n^3 - (n-1)^3\right)x^n \\
        &= \sum_{n=1}^\infty \left(3n^2 - 3n + 1\right)x^n \\
        &= 3\Gx 2 - 3\Gx 1 + \Gx 0 - 1
        \label{eq:2.1.2}
    \end{derive}

    \begin{enumerate}
        \item[$\Gx 0$:] 根据函数$f(x) = 1/(1-x)$的泰勒展开,可得$\Gx 0 = 1/(1-x)$。
        \item[$\Gx 1$:] 计算$\Gx 1 - x\Gx 1$,得到

        \begin{derive}[\Gx 1 - x\Gx 1]
            &= \sum_{n=0}^\infty nx^n- \sum_{n=1}^\infty nx^{n+1} \\
            &= \sum_{n=1}^\infty x^n \\
            &= \Gx 0 - 1 \\
            &= \frac{x}{1-x}
        \end{derive}

        因此,$\Gx 1 = \left(\Gx 0 - 1\right)/(1-x) = x/(1-x)^2$。

        \item[$\Gx 2$:] 计算$\Gx 2 - x\Gx 2$,得到

        \begin{derive}[\Gx 2]
            &= \sum_{n=0}^\infty n^2x^n- \sum_{n=1}^\infty n^2x^{n+1} \\
            &= \sum_{n=1}^\infty \left(n^2 - (n-1)^2\right)x^n \\
            &= \sum_{n=1}^\infty \left(2n - 1\right)x^n \\
            &= 2\Gx 1 - \Gx 0 + 1 \\
            &= \frac{x(1 + x)}{(1 - x)^2}
        \end{derive}

        因此,$\Gx 2 = \left\{[x(1 + x)]/[(1 - x)^2]\right\}/(1-x) = x(1+x)/(1-x)^3$。
    \end{enumerate}

    将$\Gx i, i=0,1,2$代入\eqnref{eq:2.1.2},并设各项最简分式系数为$A, B, C, D$,得到

    \begin{derive}[\Gx 3]
        &= \left[\frac{3x(1+x)}{(1-x)^3} - \frac{3x}{(1-x)^2} + \frac{1}{1-x} - 1\middle] \right/ (1 - x) \\
        &= \frac{x^3 + 4x^2 + x}{(1-x)^4} \\
        &= \frac{A}{1 - x} + \frac{B}{(1-x)^2} + \frac{C}{(1-x)^3} + \frac{D}{(1-x)^4}
    \end{derive}

    得到线性方程组

    \begin{equation}
        \begin{bmatrix}
            -1 & 0 & 0 & 0 \\
            3 & 1 & 0 & 0 \\
            -3 & -2 & -1 & 0 \\
            1 & 1 & 1 & 1 \\
        \end{bmatrix}\begin{bmatrix}
            A \\ B \\ C \\ D
        \end{bmatrix} = \begin{bmatrix}
            1 \\ 4 \\ 1 \\ 0
        \end{bmatrix}
    \end{equation}

    解得$\begin{bmatrix}A & B & C & D\end{bmatrix}^\top = \begin{bmatrix} -1 & 7 & -12 & 6 \end{bmatrix}^\top$,即

    \begin{equation}
        G(x) = -\frac{1}{1-x} + \frac{7}{(1-x)^2} - \frac{12}{(1-x)^3} + \frac{6}{(1-x)^4}
    \end{equation}

    \paragraph*{2.2} 沿用\textbf{2.1}中$\Gx n$的定义:

    \begin{derive}[G(x)]
        &= \sum_{n=0}^\infty \binom{3+n}{3} x^n \\
        &= \sum_{n=0}^\infty \frac{(3+n)!}{3!n!} x^n \\
        &= \sum_{n=0}^\infty \frac{n^3 + 6n^2 + 11n + 6}{6}x^n \\
        &= \frac 16 \Gx 3 + \Gx 2 + \frac{11}{6}\Gx 1 + \Gx 0 \\
        &= \frac{x^3 + 4x^2 + x}{6(1-x)^4} + \frac{x(1+x)}{(1-x)^3} + \frac{11x}{6(1-x)^2} + \frac{1}{1-x} \\
        &= \frac{1}{(1-x)^4}
    \end{derive}

    \paragraph*{2.5} 已知Fibonacci数列的递推公式为$F_{n+1} = F_n + F_{n-1}$。对于$G_{n + 1} = F_{2n+2}$,有

    \begin{derive}[G_{n+1}]
        &= F_{2n+2} \\
        &= F_{2n + 1} + F_{2n} \\
        &= 2F_{2n} + F_{2n - 1} \\
        &= 2F_{2n} + \left(F_{2n} - F_{2n - 2}\right) \\
        &= 3G_{n} - G_{n-1}
        \label{eq:2.5.1}
    \end{derive}
    
    得证。设序列$G_n = $对应的母函数$G(x) = G_0 + G_1 x + \cdots + G_n x^n + \cdots$,根据式\eqnref{eq:2.5.1},有

    \begin{equation}
        (1 - 3x + x^2)G(x) = G_0 + (G_1 - 3G_0)x = x
    \end{equation}

    将$G(x)$化简为最简分式,得到

    \begin{derive}[G(x)]
        &= \frac{x}{1-3x+x^2} \\
        &= \frac{x}{\left(\frac{3 + \sqrt 5}{2} - x\right)\left(\frac{3 - \sqrt 5}{2} - x\right)} \\
        &= \frac{\sqrt 5 + 1}{2x - \left(3 + \sqrt 5\right)} + \frac{1 - \sqrt 5}{2x - \left(3 - \sqrt 5\right)}
    \end{derive}

    \paragraph*{2.15}

    已知序列的母函数为$G(x) = \frac{1}{1 - x + x^2}$。则$a_0 = G(0) = 1, a_1 = G'(0) = 1$。且:

    \begin{equation}
        (x^2 - x + 1)G(x) - 1 = 0
    \end{equation}

    将$G(x)$展开为$\sum_{n=0}^\infty a_nx^n$的形式,则:

    \begin{equation}
        \begin{aligned}
            &(x^2 - x + 1)\sum_{n=0}^\infty a_nx^n - 1 = 0 \\
            \Rightarrow & \sum_{n=0}^\infty a_nx^n(x^2 - x + 1) \\
            \Rightarrow & \sum_{n=2}^\infty (a_n - a_{n-1} + a_{n-2})x^n + (a_0 - a_1)x + a_0 - 1 = 0 \\
            \Rightarrow & \sum_{n=2}^\infty (a_n - a_{n-1} + a_{n-2})x^n = 0 \\
            \Rightarrow & a_n - a_{n-1} + a_{n-2} = 0 \\
            \Rightarrow & a_n = a_{n-1} - a_{n-2}
        \end{aligned}
    \end{equation}

    \paragraph*{2.22} 已知$a_n = 3^nc + (-1)^nd$,则$a_0 = c+d, a_1 = 3c-d$。设序列$\left\{a_n\right\}$对应的母函数为$G(x) = \sum_{n=0}^\infty \left(3^nc + (-1)^nd\right)x^n$,则

    \begin{derive}[G(x) + xG(x)]
        &= \sum_{n=0}^\infty \left(3^nc + (-1)^nd\right)x^n + \sum_{n=1}^\infty \left(3^{n-1}c + (-1)^{n-1}d\right)x^n \\
        &= (c + d) + 4c\sum_{n=1}^\infty (3x)^n \\
        &= (c + d) + \frac{4cx}{1 - 3x} \\
        G(x) &= \frac{(c+d)(1-3x) + 4cx}{(1-3x)(1+x)} \\
        &= \frac{(c+d) + (c-3d)x}{(1-3x)(1+x)} \\
        (-3x^2 - 2x + 1)G(x) &= (c + d) + (c - 3d)x \\
    \end{derive}

    从而

    \begin{derive}[\sum_{n=2}^\infty (a_n - 2a_{n-1} - 3a_{n-2})x^n]
        &= (c + d) + (c - 3d)x - a_0 - (a_0 + a_1)x \\
        &= -3(c + d)x
    \end{derive}

    \paragraph*{2.49} 设排列数目为$a_n$,根据题意,有$a_0 = 0, a_1 = 0$,首先求$a_n$满足的递推关系:

    将$n$个A,B,C,D组成的序列划分为前$n-1$个元素组成的序列和最后一个元素,若一个序列满足AB出现至少一次,则可以分为两种情况:

    \begin{enumerate}
        \item AB已经在前$n-1$个元素组成的序列中出现至少一次
        \item 前$n-2$个元素中没有AB出现,且序列由AB结尾
    \end{enumerate}

    由此,可以写出$a_n$满足的递推关系为:

    \begin{equation}
        a_n - 4a_{n-1} + a_{n-2} = (n-2)!
    \end{equation}

    首先求解线性齐次递推关系$a_n - 4a_{n-1} + a_{n-2} = 0$,设该递推关系对应的母函数为$G(x)$,由递推关系,可得

    \begin{equation}
        (1 - 4x + x^2)G(x) = a_0 + (a_1 - 4a_0) = 0
    \end{equation}

    解方程$x^2 - 4x + 1 = 0$,解得

    \begin{equation}
        x = 2 \pm \sqrt{3}
    \end{equation}

    由此,
\end{document}