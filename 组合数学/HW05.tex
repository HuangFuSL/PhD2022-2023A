\documentclass{../notes}

\title{组合数学 HW05}

\begin{document}
    \maketitle

    \paragraph*{4.22}

    由于肖像的非对称性,绕立方体相对的两个面的面心旋转任意角度都不会产生不变的旋转结果(6+3阶,6个$90^\circ$、3个$180^\circ$)。

    考虑立方体绕其他轴的旋转,有

    \begin{enumerate}
        \item 不变置换(1阶)
        \item 绕立方体相对的两个顶点旋转$120^\circ$(8阶);
        \item 绕立方体相对的两条棱旋转$180^\circ$(6阶)。
    \end{enumerate}

    根据肖像的方向不同,共有四种方向,因此原问题转化为立方体的4着色问题。根据如上分析,置换群的阶数为$1+6+3+8+6=24$。则总的着色方案数为

    \begin{equation}
        \frac{1}{24}\left(4^6 + 8\times 4^2 + 6\times 4^3\right) = 192
    \end{equation}

    \paragraph*{4.24}

    \begin{subquestions}
        \item 足球的每个顶点由一个正五边形和两个正六边形拼接而成,缺角为

        \begin{equation}
        360^\circ - 108^\circ - 2\times 120^\circ = 12^\circ
        \end{equation}

        因此足球的顶点个数为$720^\circ / 12^\circ = 60$个顶点。每个顶点接一个五边形,则五边形个数为$60/5=12$个。每个顶点接两个六边形,六边形个数为$60/6\times 2=20$个。共有棱$12 \times 5 + 20 \times 3 / 2 = 90$条。

        \item 12个五边形的总着色方案数为$12!$,考虑足球的转动群。每个顶点可以通过一步旋转操作变为另一个顶点,因此转动群的阶数为$60$。

        而由于任意两个五边形的着色颜色不相同,因此经过任何一次旋转,都不会产生不变的旋转结果:只有不变置换可以产生不变旋转结果。因此,总的方案数为$12!/60$种。
    \end{subquestions}

    \paragraph*{4.28} 原问题相当于同时对顶点和面2着色。正八面体的对称性与立方体相同。根据\textbf{4.22},有正八面体转动群的阶数为$24$。

    \begin{enumerate}
        \item 不变置换
        \item 绕顶点旋转$90^\circ$,共有6个
        \item 绕顶点旋转$180^\circ$,共有3个
        \item 绕相对的两面面心旋转$120^\circ$,共有8个
        \item 绕相对的两条棱心旋转$180^\circ$,共有6个
    \end{enumerate}

    考察每种置换中的不动点(不发生改变的八面体)数量

    \begin{enumerate}
        \item 对于不变置换,面的染色方案有$\binom{8}{4}$种,顶点的染色方案有$\binom{6}{4}$种
        \item 当绕顶点旋转$90^\circ$时,八面体的面可以分为两组相互等价的面,每组4个;顶点可以分为两组相互等价的顶点,分别有1、4、1个。相互等价的面(顶点)需要染上相同的颜色(下同)。因此经过变换不发生改变的八面体中,顶点的染色方案只有一种;面的染色方案有两种。
        \item 当绕顶点旋转$180^\circ$时,八面体的面可以分为四组相互等价的面,每组2个;顶点可以分为四组相互等价的顶点,分别有1、2、2、1个。经过变换不发生改变的八面体中,面的染色方案有$\binom{4}{2}$种,顶点的染色方案有$3$种。
        \item 当绕相对的两面面心旋转$120^\circ$时,八面体的面可以分为四组相互等价的面,分别有1、3、3、1个。顶点可以分为两组相互等价的顶点,每组3个。由于顶点其中的两个为蓝色,四个为红色,因此不存在合适的不动点。
        \item 当绕相对的两条棱心旋转$180^\circ$时,八面体的面可以分为四组相互等价的面,每组2个;顶点可以分为三组相互等价的顶点,每组3个。因此经过变换不发生改变的八面体中,面的染色方案有$\binom{4}{2}$种,顶点的染色方案有$\binom{3}{2}$种。
    \end{enumerate}

    因此总的染色方案数为

    \begin{equation}
        \frac{1}{24}\left(\binom{6}{4}\binom{8}{4} + 6\times 1\times 2 + 3\times 3\times \binom{4}{2} + 6\times \binom{4}{2}\binom{3}{2}\right) = 51
    \end{equation}
\end{document}