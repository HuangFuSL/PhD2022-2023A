\documentclass[UTF8]{ctexart}
\usepackage{amsmath}
\usepackage{amssymb}
\usepackage{enumitem}

\newcommand{\pe}[2]{P_{#1}^{#2}}
\newcommand{\co}[2]{C_{#1}^{#2}}

\newcommand{\N}{\mathbb N}
\newcommand{\R}{\mathbb R}
\newcommand{\Z}{\mathbb Z}
\newcommand{\C}{\mathbb C}
\newcommand{\dd}{\mathrm d}
\renewcommand{\geq}{\geqslant}
\renewcommand{\leq}{\leqslant}
\newcommand{\eqnref}[1]{(\ref{#1})}
\newcommand{\prob}[1]{P\left(#1\right)}
\newcommand{\cprob}[2]{P\left(#1|#2\right)}
\newcommand{\bs}{\boldsymbol}

\title{HW01}
\author{皇甫硕龙}

\begin{document}
    \maketitle
    \paragraph*{1.3}

    \begin{enumerate}[label=\textbf{(\arabic*)}, itemsep=0.2em]
        \item 男生不相邻,相当于把$m$个男生放置$n$个女生组成的$n+1$空隙中,则为$\pe{n+1}{m}$,由于女生排列存在顺序,总的排列数为$\pe nn\pe{n+1}{m}$
        \item 将女生视为整体后的排列数为$\pe{m+1}{m+1}$,女生内部有$\pe{n}{n}$种排列,因此总排列数为$\pe{m+1}{m+1}\pe{n}{n}$
        \item 将男生A和女生B视为整体,排列数为$\pe{m+n-1}{m+n-1}$,A和B有$\pe 22$种排列,因此总排列数为$\pe 22\pe{m+n-1}{m+n-1}$
    \end{enumerate}

    \paragraph*{1.8}

    已知$10^{40} = 2^{40}\times 5^{40}, 20^{30} = 2^{60}\times 5^{30}$,因此两者的公因数可以写作$2^i5^j, 0\leq i\leq 40, 0\leq j\leq 30$,因此共有$41\times 31 = 1271$个公因数。

    \paragraph*{1.16} 使用隔板法,组合数为$\prod _{i=n-1}^{r-1}i = \co{n-1}{r-1}$
    \paragraph*{1.19} 使用隔板法,1作为隔板分割0,可以从两端开始,因此组合数为$\co{m+1}{n}$
    \paragraph*{1.27} 
    
    \begin{enumerate}[label=\textbf{(\arabic*)}, itemsep=0.2em]
        \item 女宾不相邻时,5位女宾周围共有5个空隙,需要放入6位男宾。不考虑男宾时,方案数为$\pe 55/5$,考虑男宾,男宾共有$\pe 66$种排列顺序,因此总的排列数为$\pe 55\pe 66/5$
        \item 所有女宾在一起时,方案数为$\pe 55\pe 77 / 7 = \pe 55\pe 66$
        \item 将A和两位男宾视为一个整体,方案数为$\pe 99 / 9$,女宾和周围两位男宾共有$\pe 62$种排列方法,因此总的方案数位$\pe 62\pe 99 / 9 = \pe 62\pe 88$
    \end{enumerate}

    \paragraph*{1.33} 每盒至少$k$个球,相当于移去$n(k - 1)$个球后每盒至少$1$个球,则总的排列数为$\co{r - n(k-1) - 1}{n - 1}$

    \paragraph*{1.45} 
    
    \begin{enumerate}[label=\textbf{(\arabic*)}, itemsep=0.2em]
        \item 设取$i$个互不相同的球,则可取$n - i$个相同的球
        
        \begin{equation}
            \sum_{i=0}^{n} \co{n}{i} = 2^n
        \end{equation}

        \item 同上,结果为
        
        \begin{equation}
            \sum_{i=0}^{n} \co{2n+1}i = 2^{2n}
        \end{equation}

    \end{enumerate}
    
    \paragraph*{1.48} 由题,格路法到达$x + y = k (x\geq y)$的点的总方案数量为

    \begin{equation}
        \sum_{i=1}^{p} \co ki
    \end{equation}

    式中$p$为$x + y = k$落在区域$x\geq 0, y\leq n, y\geq x$内的整数点的数量。

    \begin{enumerate}
        \item[$k\leq n$时] 有$p = \left\lfloor k/2 \right\rfloor$
        \item[$k> n$时] 有$p = n - \left\lfloor k/2 \right\rfloor$ 
    \end{enumerate}

\end{document}