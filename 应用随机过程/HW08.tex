\documentclass{../notes}

\title{应用随机过程 HW08}

\begin{document}
    \maketitle

    \paragraph*{6.2}

    \begin{subquestions}
        \item[$E(X(t))$] 首先计算$\bs Q$的特征值,解方程

        \begin{equation}
            \begin{vmatrix}
                -\lambda - t & \lambda \\
                \mu & -\mu - t
            \end{vmatrix} = 0
        \end{equation}

        解得$t_1 = 0, t_2 = -(\lambda + \mu)$,对应的特征向量为

        \begin{equation}
            \begin{aligned}
                x_1 = \begin{bmatrix}
                    1 \\ 1
                \end{bmatrix} &
                x_2 = \begin{bmatrix}
                    \lambda \\ -\mu
                \end{bmatrix}
            \end{aligned}
        \end{equation}

        令矩阵$\bs A = \begin{bmatrix} x_1 & x_2 \end{bmatrix}$,则

        \begin{derive}[\bs Q^n]
            &= \bs A\diag(0, -(\mu + \lambda))^n \bs A^{-1} \\
            &= \frac{1}{\lambda+\mu}\begin{bmatrix}
                1 & \lambda \\
                1 & -\mu
            \end{bmatrix}\begin{bmatrix}
                0 & 0 \\
                0 & (-1)^n (\mu+\lambda)^n
            \end{bmatrix}\begin{bmatrix}
                \mu & 1 \\
                -\lambda & -1
            \end{bmatrix} \\
            &= \frac{(-1)^n(\lambda + \mu)^n}{\lambda + \mu}\begin{bmatrix}
                -\lambda & \lambda \\
                \mu & -\mu \\
            \end{bmatrix}
        \end{derive}

        因此,$\bs P(t) = \exp\left(\bs Qt\right)$,即

        \begin{derive}[\bs P(t)]
            &= e^{\bs Qt} \\
            &= \sum_{n=0}^\infty \frac{t^n}{n!} \bs Q^n \\
            &= \sum_{n=0}^\infty \frac{(-1)^n(\lambda + \mu)^n t^n}{(\lambda + \mu)n!}\begin{bmatrix}
                -\lambda & \lambda \\
                \mu & -\mu \\
            \end{bmatrix} \\
            &= \frac{1}{{\lambda + \mu}} \left(\begin{bmatrix}
                \mu & \lambda \\
                \mu & \lambda
            \end{bmatrix} + e^{-(\lambda + \mu)t}\begin{bmatrix}
                \lambda & -\lambda \\
                -\mu & \mu \\
            \end{bmatrix}\right)
        \end{derive}

        $X(t)$的分布为$\bs \pi = \bs \pi_0 \bs P(t) = \frac{1}{\lambda + \mu} \left(\begin{bmatrix} \mu & \lambda \end{bmatrix} + e^{-(\lambda + \mu)t} \begin{bmatrix} \lambda & -\lambda\end{bmatrix}\right)$,因此

        \begin{equation}
            E(X(t)) = \frac{1}{\lambda + \mu}\left(\lambda -\lambda e^{-(\lambda + \mu)t}\right)
        \end{equation}

        \item[$E(\tau_1|X(0) = 0)$] 已知$P(\tau_1\geq t|X(0) = 0) = e^{q_{00}t} = e^{-\lambda t}$,设概率分布函数$F(\tau_1|X(0) = 0)$,对应的概率密度函数为$f(\tau_1|X(0) = 0)$,则

        \begin{equation}
            f(\tau_1|X(0) = 0) = \frac{\dd F}{\dd t} = \lambda e^{-\lambda t}
        \end{equation}

        因此,$E(\tau_1|X(0) = 0)$为:

        \begin{derive}[E(\tau_1|X(0) = 0)]
            &= \int_0^\infty \lambda \tau_1 e^{-\lambda \tau_1} \dd \tau_1 \\
            &= \frac{1}{\lambda}
        \end{derive}
    \end{subquestions}

    \paragraph*{6.5 (3)} 设$t$时刻时系统处于状态$n$,取一个极短的时间$\Delta t$,根据生灭过程的性质,有

    \begin{equation}
        \begin{aligned}
            P(X(t+\Delta t) = n+1|X(t) = n) &= (n\lambda +a)\Delta t \\
            P(X(t+\Delta t) = n-1|X(t) = n) &= n\mu \Delta t \\
            P(X(t+\Delta t) = n|X(t) = n) &= 1 - (n\lambda + n\mu + a)\Delta t \\
        \end{aligned}
    \end{equation}

    记$p_n(t) = P(X(t) = n)$,根据全概率公式,有

    \begin{equation}
        \begin{aligned}
            p_0(t+\Delta t) &= (1 - n\mu \Delta t)p_0(t) + n\mu \Delta tp_1(t) \\
            p_n(t+\Delta t) &= ((n-1)\lambda +a)\Delta tp_{n-1}(t) + \left[1 - (n\lambda + n\mu + a)\Delta t\right]p_n(t) + (n+1)\mu \Delta tp_{n+1}(t)
        \end{aligned}
    \end{equation}

    由此,有

    \begin{derive}[\frac{\dd p_j(t)}{\dd t}]
        &= \lim_{\Delta t\rightarrow 0}\frac{p_n(t+\Delta t) - p_n(t)}{\Delta t} \\
        &= ((n-1)\lambda +a)p_{n-1}(t) - (n\lambda + n\mu + a)p_n(t) + (n+1)\mu p_{n+1}(t)
    \end{derive}

    当系统趋向于稳态时,设$p_n = \lim_{t\rightarrow \infty}p_n(t)$。有$\lim _{t\rightarrow\infty} \frac{\dd p_n(t)}{\dd t} = 0$,即

    \begin{equation}
        ((n-1)\lambda +a)p_{n-1} - (n\lambda + n\mu + a)p_n + (n+1)\mu p_{n+1}
    \end{equation}

    解得

    \begin{equation}
        p_{n+1} = \frac{(n\lambda + n\mu + a)p_n - ((n-1)\lambda + a)p_{n-1}}{(n+1)\mu}
    \end{equation}

    同时,对于$n=0$,有$p_1 = \frac{a}{\mu}p_0$。设序列$a_n$满足$p_n = a_np_0$,则有

    \begin{equation}
        \left\{
        \begin{aligned}
            a_0 &= 1 \\
            a_1 &= \frac{a}{\mu} \\
            a_{n+1} &= \frac{(n\lambda + n\mu + a)a_n - ((n-1)\lambda + a)a_{n-1}}{(n+1)\mu}
        \end{aligned}
        \right.
    \end{equation}

    从而

    \begin{equation}
        a_n = \prod_{i=1}^n \frac{(i-1)\lambda + a}{i\mu} = \frac{1}{n! \mu^n}\prod_{i=1}^n \left[(i-1)\lambda + a\right]
    \end{equation}

    设$S = \sum_{n=0}^\infty a_n$,则$p_0 = 1/S$:

    \begin{derive}[S]
        &= \sum_{n=0}^\infty a_n \\
        &= 1 + \sum_{n=1}^\infty \left(\frac{1}{n!\mu^n}\right)\prod_{i=1}^n\left[(i-1)\lambda + a\right]
    \end{derive}

    首先证明$S$收敛,有

    \begin{derive}[S]
        &\leq 1 + \sum_{n=1}^\infty \frac{1}{n!\mu^n} \prod_{i=1}^n\left[(i-1)\lambda + (i-1)a\right] \\
        &=1 + \frac{1}{\lambda + a}\sum_{n=1}^\infty \frac{(\lambda + a)^n}{n\mu^n} \\
        &= 1 + \frac{1}{\lambda + a}\sum_{n=1}^\infty \frac{1}{n}\frac{(\lambda + a)}{\mu} \\
        &= 1 + \frac{1}{\lambda + a}\log \frac{\mu}{\mu - \lambda - a}
    \end{derive}

    且

    \begin{equation}
        S = \left(1-\frac{\lambda }{\mu }\right)^{-\frac{a}{\lambda }}
    \end{equation}

    因此,有

    \begin{equation}
        p_{n} = a_{n}p_0 = \left(1-\frac{\lambda }{\mu }\right)^{\frac{a}{\lambda }}\left(\prod_{i=1}^n \frac{(i-1)\lambda + a}{i\mu}\right) \label{eq:6.5.1}
    \end{equation}

    \paragraph*{6.9}

    \begin{subquestions}
        \item 对于$M/M/s$队列,有

        \begin{equation}
            \left\{
            \begin{aligned}
                \lambda_i &= \lambda \\
                \mu_i &= \mu \min\left\{i, s\right\}
            \end{aligned}
            \right.
        \end{equation}

        与\textbf{6.5 (3)}同理,有

        \begin{derive}[a_n]
            &= \frac{\lambda_0\lambda_1\cdots\lambda_{n-1}}{\mu_1\mu_2\cdots\mu_n} \\
            &= \left\{
            \begin{aligned}
                & \frac{s^s\rho^n}{s!} & n > s \\
                & \frac{s^n\rho^n}{n!} & 1\leq n \leq s
            \end{aligned}
            \right.
        \end{derive}

        令$S = \sum_{n=0}^\infty a_n$,则$p_0 = 1/S$:

        \begin{derive}[S]
            &= \sum_{n=0}^\infty a_n \\
            &= \sum_{n=0}^s a_n + \sum_{n=s+1}^\infty a_n \\
            &= \sum_{n=0}^s \frac{s^n\rho^n}{n!} + \sum_{n=s+1}^\infty \frac{s^s\rho^n}{s!} \\
            &= \sum_{n=0}^s \frac{s^n\rho^n}{n!} + \frac{\rho}{1-\rho}\frac{(\rho s)^s}{s!}
        \end{derive}

        因此,平稳分布为

        \begin{equation}
        \left\{
        \begin{aligned}
            p_0 &= 1/S & \\
            p_n &= \left.\frac{s^n\rho^n}{n!} \right/S & 1\leq n\leq s \\
            p_n &= \left.\frac{s^s\rho^n}{s!} \right/S & n > s \\
        \end{aligned}
        \right.
        \end{equation}

        \item 首先计算$P(Q(t) = 0)$,即$n\leq s$,有

        \begin{derive}[P(Q(t) = 0)]
            &= \frac{1}{S}\sum_{n=0}^s a_n \\
            &= \frac{1}{S}\sum_{n=0}^s \frac{s^n\rho^n}{n!} \\
        \end{derive}

        其次计算$E[Q(t)]$,首先考虑$P(Q(t) = n)$,其中$s > 0$,有

        \begin{equation}
            P(Q(t) = n) = \frac{1}{S}\frac{s^s\rho^n}{s!}
        \end{equation}

        因此,$E(Q(t))$为

        \begin{derive}[E(Q(t))]
            &= \sum_{n={s+1}}^\infty (n-s)\frac{1}{S}\frac{s^s\rho^n}{s!} \\
            &= \frac{s^s}{s!S}\left[-s\sum_{n=s+1}^\infty \rho^n + \sum_{n=s+1}^\infty n\rho^n\right] \\
            &= \frac{s^s}{s!S}\left[-\frac{s\rho^s}{1-\rho} + \frac{\rho^s(1+s-s\rho)}{(1-\rho)^2}\right] \\
            &= \frac{s^s\rho^s}{s!S}\frac{1}{(1-\rho)^2}
        \end{derive}

        设$P(Q(t) = 0) = r$

        \begin{derive}[1-r]
            &= 1 - \frac{1}{S}\sum_{n=0}^s \frac{s^n\rho^n}{n!} \\
            &= \frac{s^s\rho^s}{s!S}\frac{1}{1-\rho}
        \end{derive}

        因此,有$E(Q(t)) = (1-r)(1-\rho)^{-1}$
    \end{subquestions}

    \paragraph*{6.15}

    \begin{subquestions}
        \item $P(X(t) = i|X(t)\in B) = P(X(t) = i) / P(X(t) \in B)$,即

        \begin{equation}
            P(X(t) = i|X(t)\in B) = \frac{P_i}{\sum_{j\in B}P_j}
        \end{equation}

        \setcounter{enumi}{2}
        \item 根据$\tilde F_i(s)$的定义,有

        \begin{derive}[\tilde F_i(s)]
            &= E\{e^{-ST_i} | X(0) = i\} \\
            &= \sum_{j\in S} P\left(X(\alpha) = j|X(0) = i\right)E\left\{e^{-ST_i}\middle | X(0) = i, X(\alpha) = j\right\}
        \end{derive}

        式中$\alpha = \inf\{t: t > 0, X(0) = i, X(\alpha) \not = i\}$,则

        \begin{derive}[\tilde F_i(s)]
            =& \sum_{j\in S} P_{ij}E\left\{e^{-ST_i}\middle | X(0) = i, X(\alpha) = j\right\} \\
            =& \sum_{j\in S}  P_{ij}E\left\{e^{-S\alpha}\middle | X(0) = i\right\}E\left\{e^{-ST_j}\middle | X(0) = j\right\} \\
            =& \sum_{j\in G}  P_{ij}E\left\{e^{-S\alpha}\middle | X(0) = i\right\}E\left\{e^{-ST_j}\middle | X(0) = j\right\} + \\
                &\sum_{j\in B} P_{ij}E\left\{e^{-S\alpha}\middle | X(0) = i\right\}E\left\{e^{-ST_j}\middle | X(0) = j\right\} \\
            =& \sum_{j\in G} P_{ij} \frac{q_i}{q_i + s} + \sum_{j\in B} P_{ij} \frac{q_i}{q_i + s}\tilde{F}_j(s) \\
            =& \frac{q_i}{q_i + s}\left(\sum_{j\in B} P_{ij}\tilde{F}_j(s) + \sum_{j\in G}P_{ij}\right)
        \end{derive}
    \end{subquestions}

    \paragraph*{6.16} 由向前方程,有

    \begin{align}
        P'_{ij}(t) &= P_{i, j-1}(t)\lambda _{j-1} - P_{ij}(t)\lambda j & (j\geq i) \label{eq:6.16.1} \\
        P_{ii}(0) &= 1 \\
        P_{ij}(0) &= 0 & (j > i)
    \end{align}

    令$j = i$,解得

    \begin{equation}
        P_{ii}(t) = e^{-\lambda i + \delta}t
    \end{equation}

    解方程\eqnref{eq:6.16.1}可得

    \begin{equation}
        P_{i, j+1}(t) e^{\lambda_{j+1}t} = \lambda_j \int_{0}^t P_{ij}(u)e^{\lambda_{j+1}u}\dd u
    \end{equation}

    由$P_{ii}(0) = 1, P_{ij}(0) = 0$,可得$j > i$时的$P_{ij}(t)$

    \begin{equation}
        P_{ij}(t) = \frac{1}{(j-i)!}\prod_{k=0}^{j-(i+1)}\left(k+i+\frac{\delta}{\lambda}\right)e^{-(\lambda i + \delta)t}\left(1 - e^{-\lambda t}\right)^{j-i}
    \end{equation}

    \paragraph*{6.17}

    \begin{subquestions}
        \item 当$\lambda > \mu$时,显然有

        \begin{equation}
            \lambda_n = n\lambda + \delta > n\mu = \mu_n \Rightarrow \sum_{n=1}^\infty \prod_{i=1}^n \frac{\mu_i}{\lambda i} < \sum_{n=1}^\infty \left(\frac{\mu}{\lambda}\right)^n < \infty
        \end{equation}

        当$\lambda = \mu < \delta$时,有

        \begin{equation}
            \sum_{n=1}^\infty \prod_{i=1}^n \frac{\mu_i}{\lambda i} = \sum_{n=1}^\infty \prod_{i=1}^n \frac{i\lambda}{i\lambda + \delta} \label{eq:6.17.1}
        \end{equation}

        考察\eqnref{eq:6.17.1}的第$n$项:

        \begin{equation}
            a_{n} = \prod_{i=1}^n \frac{i\lambda}{i\lambda \delta}
        \end{equation}

        $a_n$满足$a_n / a_{n+1} = \frac{(n+1)\lambda + \delta}{(n+1)\lambda}$,则

        \begin{equation}
            \lim_{n\rightarrow \infty} n\left(\left|\frac{a_{n}}{a_{n+1}} - 1\right|\right) = \lim_{n\rightarrow \infty} \frac{n\delta}{(n+1)\lambda} = \frac{\delta}{\lambda} > 1
        \end{equation}

        因此$\sum a_n < \infty$。在以上两种情况下,有$\sum_{n=1}^\infty \prod_{i=1}^n \frac{\mu_i}{\lambda i} < \infty$,从而$\sum_{n=1}^\infty \prod_{i=1}^n \frac{\tilde \mu_i}{\tilde \lambda i}$,即生灭过程的嵌入链非常返,从而该生灭过程非常返。

        \item 当$\lambda = \mu \geq \delta$时,正项级数$\sum a_n$发散,因此嵌入链常返。设$\lambda = \mu = k\delta, k\geq 1$,则

        \begin{derive}[\sum_{n\geq 1}^\infty \prod_{i=1}^{n} \frac{(i-1)\lambda + \delta}{i\mu}]
            &= \sum_{n\geq 1}^\infty \prod_{i=1}^n \frac{1}{i}\left((i - i) + \frac{1}{k}\right) \\
            &= \frac{1}{k}\sum_{n\geq 1}^\infty \frac{1}{n} \prod_{i=1} ^{n-1} \left(1 + \frac{1}{ik}\right) \\
            &\geq \frac{1}{k} \sum_{n=1}^\infty \frac{1}{n} = \infty
        \end{derive}

        因此不存在平稳分布,马氏链为零常返

        \item 当$\lambda < \mu$时,根据式\eqnref{eq:6.5.1},有

        \begin{equation}
        \left\{
        \begin{aligned}
            p_{0} &= \left(1 - \frac{\lambda}{\mu}\right)^{\frac{\delta}{\lambda}} \\
            p_n &= \left(1 - \frac{\lambda}{\mu}\right)^{\frac{\delta}{\lambda}}\frac{\delta(\lambda + \delta)\cdots\left[(n-1)\lambda + \delta\right]}{n!\mu^n}
        \end{aligned}
        \right.
        \end{equation}
    \end{subquestions}
\end{document}