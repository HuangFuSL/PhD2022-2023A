\documentclass{../notes}

\title{应用随机过程 HW07}

\newcommand{\Gx}[1]{G^{(#1)}(x)}

\begin{document}
    \maketitle

    \paragraph*{2.14}

    $N(1)$服从如下分布:

    \begin{equation}
        \begin{aligned}
            \prob{N(1) = 0} &= \frac{2}{3} \\
            \prob{N(1) = 1} &= \frac{1}{3} \\
        \end{aligned}
    \end{equation}

    $N(2)$服从如下分布:

    \begin{equation}
        \begin{aligned}
            \prob{N(2) = 1} &= \frac{8}{9} \\
            \prob{N(2) = 2} &= \frac{1}{9} \\
        \end{aligned}
    \end{equation}

    $N(3)$服从如下分布:

    \begin{equation}
        \begin{aligned}
            \prob{N(3) = 1} &= \frac{4}{9} \\
            \prob{N(3) = 2} &= \frac{14}{27} \\
            \prob{N(3) = 3} &= \frac{1}{27} \\
        \end{aligned}
    \end{equation}

    \paragraph*{2.15}

    $\lim _{n\rightarrow\infty} (EN_x / x) = 1 / \mu = 1 / (EX + EY)$,由于$Y\sim \mathcal U[0, 1]$,有$EY = 1/2$,已知$x$的分布函数$F(x) = 1 - e^{-x}$,有

    \begin{derive}[EX]
        &= \int_{0}^\infty x\dd F(x) \\
        &= \int_{0}^\infty xe^{-x} \dd x \\
        &= 1
    \end{derive}

    因此$\lim_{n\rightarrow\infty} (EN_x / x) = 2 / 3$

    \paragraph*{2.17}

    设$X$的分布函数为$F(x)$,对$F$作Laplace变换,有

    \begin{derive}[\tilde{F} (s)]
        &= \int_{0}^\infty e^{-st}\dd F(t) \\
        &= \frac{\lambda ^2}{(\lambda + s)^2} \\
    \end{derive}

    由此,更新函数的Laplace变换即为

    \begin{equation}
        \tilde{m}(s) = \frac{\tilde F(s)}{1 - \tilde F(s)} = \frac{\lambda ^2}{(\lambda + s)^2 - \lambda^2}
    \end{equation}

    得到更新函数为

    \begin{equation}
        m'(t) = \frac{1}{2} \lambda \left(1 - e^{-2 \lambda t}\right)
    \end{equation}

    积分,得到

    \begin{derive}[m(t)]
        &= \int_{0}^t m'(s)\dd s \\
        &= \left(\frac{e^{-2 \lambda s}}{4}+\frac{1}{2}\lambda\middle)\right|^{t}_{0} \\
        &= \frac{2\lambda t + e^{-2\lambda t}- 1}{4}
    \end{derive}

    \paragraph*{6.1} 由于$\{N(t), t\geq 0\}$为泊松过程,则$N(s+t) - N(s) \sim \text{P}(\lambda t)$,即$N(s+t) - N(s)$服从泊松分布,因此:

    \begin{align}
        \prob{N(s+t) - N(s) = 2k} &= \sum_{k=0}^\infty \frac{(\lambda t)^{2k}}{(2k)!}e^{-\lambda t} = \frac{1 + e^{-2\lambda t}}{2} \\
        \prob{N(s+t) - N(s) = 2k + 1} &= \sum_{k=0}^\infty \frac{(\lambda t)^{2k + 1}}{(2k + 1)!}e^{-\lambda t} = \frac{1 - e^{-2\lambda t}}{2}
    \end{align}

    由状态$X_t$转移到$X_{t+s}$的概率为

    \begin{equation}
        \left\{
        \begin{aligned}
            \cprob{X_{n+t} = 0}{X_n = 0} &= \frac{1 + e^{-2\lambda t}}{2} \\
            \cprob{X_{n+t} = 1}{X_n = 0} &= \frac{1 - e^{-2\lambda t}}{2} \\
            \cprob{X_{n+t} = 0}{X_n = 1} &= \frac{1 - e^{-2\lambda t}}{2} \\
            \cprob{X_{n+t} = 1}{X_n = 1} &= \frac{1 + e^{-2\lambda t}}{2} \\
        \end{aligned}
        \right.
    \end{equation}

    组织成矩阵形式,即

    \begin{equation}
        \bs P(t) = \begin{bmatrix}
            \frac{1 - e^{-2\lambda t}}2 & \frac{1 - e^{-2\lambda t}}2 \\
            \frac{1 + e^{-2\lambda t}}2 & \frac{1 + e^{-2\lambda t}}2 \\
        \end{bmatrix}
    \end{equation}

    转移率矩阵$\bs Q$:

    \begin{equation}
        \bs Q = \bs P'(0) = \begin{bmatrix}
            -\lambda & \lambda \\
            \lambda & -\lambda \\
        \end{bmatrix}
    \end{equation}

    \paragraph*{6.26}

    \subparagraph*{“$\Rightarrow$”} 已知

    \begin{equation}
        \int _0^\infty P_{ii}(t)\dd t = \infty, \forall i\in S
    \end{equation}

    由C-K方程,$\forall (n-1)h\leq t \leq nh$,即$nh-t\in [0, h]$有

    \begin{derive}[P_{ii}(nh)]
        &= \sum_{k\in S}P_{ik}(t) P_{ki}(nh - t) \\
        &\geq P_{ii}(t) P_{ii}(nh-t)
    \end{derive}

    由于$P_{ii}(x) > 0$且在$[0, +\infty)$上一致连续,则$\exists x_0\in [0, h], \forall x\in [0, h], P_{ii}(x) \geq P_{ii}(x_0) = \alpha$,即$P_{ii}(x)$存在最小值。则$P_{ii}(nh) \geq P_{ii}(t) P_{ii}(nh-t) \geq P_{ii}(t) \alpha$,对$t$积分,得到:

    \begin{equation}
        \begin{aligned}
            & \int_{nh-t}^{nh} P_{ii}(nh)\dd t \geq \alpha \int_{nh-t}^{nh} P_{ii}(t)\dd t \\
            \Rightarrow & hP_{ii}(nh) \geq \alpha \int_{nh-t}^{nh} P_{ii}(t)\dd t \\
            \Rightarrow & \sum_{n=1}^\infty hP_{ii}(nh) \geq \alpha\sum_{1}^{\infty}P_{ii}(t)\dd t = \alpha\int_{0}^\infty P_{ii}(t)\dd t = \infty \\
            \Rightarrow & \sum_{n=1}^\infty P_{ii}(nh) = \infty
        \end{aligned}
    \end{equation}

    \subparagraph*{“$\Leftarrow$”} 已知

    \begin{equation}
        \sum_{n=1}^\infty P_{ii}(nh) = \infty
    \end{equation}

    设$nh\leq t\leq (n+1)h$,则有

    \begin{derive}[P_{ii}(t)]
        &= \sum_{k\in S}P_{ik}(t-nh)P_{ki}(nh) \\
        &\geq P_{ii}(t-nh)P_{ii}(nh) \\
        &\geq \alpha P_{ii}(nh)
    \end{derive}

    对$t$积分,得到

    \begin{equation}
        \begin{aligned}
            & \int_{(n-1)h}^{nh} P_{ii}(t)\dd t \geq \int_{(n-1)h}^{nh} \alpha P_{ii}(nh)\dd t \\
            \Rightarrow & \sum_{n=1}^\infty \int_{(n-1)h}^{nh} P_{ii}(t)\dd t \geq \sum_{n=1}^\infty \int_{(n-1)h}^{nh} \alpha P_{ii}(nh)\dd t \\
            \Rightarrow & \int_{0}^\infty P_{ii}(t)\dd t\geq \sum_{n=1}^\infty h\alpha P_{ii}(nh) = \infty \\
            \Rightarrow & \int_{0}^\infty P_{ii}(t)\dd t = \infty
        \end{aligned}
    \end{equation}
\end{document}