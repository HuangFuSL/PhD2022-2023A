\documentclass{../notes}

\title{应用随机过程 HW07}

\newcommand{\Gx}[1]{G^{(#1)}(x)}

\begin{document}
    \maketitle

    \paragraph*{2.14}

    $N(1)$服从如下分布:

    \begin{equation}
        \begin{aligned}
            \prob{N(1) = 0} &= \frac{2}{3} \\
            \prob{N(1) = 1} &= \frac{1}{3} \\
        \end{aligned}
    \end{equation}

    $N(2)$服从如下分布:

    \begin{equation}
        \begin{aligned}
            \prob{N(2) = 1} &= \frac{8}{9} \\
            \prob{N(2) = 2} &= \frac{1}{9} \\
        \end{aligned}
    \end{equation}

    $N(3)$服从如下分布:

    \begin{equation}
        \begin{aligned}
            \prob{N(3) = 1} &= \frac{4}{9} \\
            \prob{N(3) = 2} &= \frac{14}{27} \\
            \prob{N(3) = 3} &= \frac{1}{27} \\
        \end{aligned}
    \end{equation}

    \paragraph*{2.15}

    $\lim _{n\rightarrow\infty} (EN_x / x) = 1 / \mu = 1 / (EX + EY)$,由于$Y\sim \mathcal U[0, 1]$,有$EY = 1/2$,已知$x$的分布函数$F(x) = 1 - e^{-x}$,有

    \begin{derive}[EX]
        &= \int_{0}^\infty x\dd F(x) \\
        &= \int_{0}^\infty xe^{-x} \dd x \\
        &= 1
    \end{derive}

    因此$\lim_{n\rightarrow\infty} (EN_x / x) = 2 / 3$

    \paragraph*{2.17} 首先计算$\Gx p = \int x^pe^{-x}\dd x, p\in \N$,有

    \begin{equation}
        \Gx p = \int x^pe^{-x}\dd x = -\left(x^p e^{-x} -p\int x^{p-1}e^{-x}\right) = p\Gx {p-1} - x^p{e^{-x}}
    \end{equation}

    又有$\Gx 0 = -e^{-x}$,从而

    \begin{equation}
        \Gx p = -p!e^{-x}\sum_{i=0}^p \frac{x^i}{i!}
    \end{equation}
    
    设$X$的分布函数为$F(x)$,对$F$作Laplace变换,有

    \begin{derive}[\tilde{F} (s)]
        &= \int_{0}^\infty e^{-st}\dd F(t) \\
        &= \frac{\lambda ^2}{(\lambda + s)^2} \\
    \end{derive}

    由此,更新函数的Laplace变换即为

    \begin{equation}
        \tilde{m}(s) = \frac{\tilde F(s)}{1 - \tilde F(s)} = \frac{\lambda ^2}{(\lambda + s)^2 - \lambda^2}
    \end{equation}

    得到更新函数为

    \begin{equation}
        m(t) = -\frac{1}{2} \lambda \left(-1 + e^{-2 \lambda t}\right)
    \end{equation}

    \paragraph*{6.1}

    \paragraph*{6.26}
\end{document}