\documentclass{../notes}

\title{应用随机过程 HW06}

\loadgeometry{word-moderate}

\begin{document}
    \maketitle

    \paragraph*{2.5} 已知$S_1, S_2, \dots, S_n$的条件概率密度为

    \begin{equation}
        f(t_1, t_2, \dots, t_n | N(t) = n) = \begin{cases}
            \frac{n!}{t^n} & 0 < t_1 < t_2 < \cdots < t_n \leq t \\
            0 & \text{otherwise} \\
        \end{cases}
    \end{equation}

    因此,对某个给定的$k\leq n$,有

    \begin{derive}[f(t_k|N(t) = n)_{t_i, i\not = k}]
        &= \int_0^{t_2}\cdots\int_0^{t_k} \int_{t_k} ^t \cdots \int_{t_{n-1}} ^t \frac{n!}{t^n} \dd t_n \cdots \dd t_{k+1} \dd t_{k-1} \cdots \dd t_1 \\
        &= \int_0^{t_2}\cdots\int_0^{t_k} \int_{t_k} ^t \cdots \int_{t_{n-2}} ^t \frac{n!(t - t_{n-1})}{t^n} \dd t_{n-1} \cdots \dd t_{k+1} \dd t_{k-1} \cdots \dd t_1 \\
        &= \int_0^{t_2}\cdots\int_0^{t_k} \int_{t_k} ^t \cdots \int_{t_{n-3}} ^t \frac{n!(t - t_{n-2})^2}{2!\times t^n} \dd t_{n-2} \cdots \dd t_{k+1} \dd t_{k-1} \cdots \dd t_1 \\
        &\vdots \\
        &= \frac{n!(t-t_k)^{n-k}}{(n-k)!t^n} \int_{0}^{t_2}\cdots \int_0^{t_k} \dd t_{k-1} \cdots \dd t_1 \\
        &= \frac{n! t_k^k (t - t_k)^{n-k}}{(n-k)!t^n} \int_{0}^{t_2}\cdots \int_0^{t_{k-1}} t_{k-2} \dd t_{k-2} \cdots \dd t_1 \\
        &\vdots \\
        &= \frac{n! t_k^{k-1} (t - t_k)^{n-k}}{(k-1)!(n-k)!t^n}
    \end{derive}

    从而

    \begin{derive}[\cexpt{S_k}{N(t) = n}]
        &= \int_{0}^{t} t_k f(t_k|N(t) = n)_{t_i, i\not = k} \dd t_{k} \\
        &= \frac{n!}{(k-1)!(n-k)!t^n}\int_{0}^{t} t_k^k (t - t_k)^{n-k} \dd t_{k} \\
        &= \frac{n!t}{(k-1)!(n-k)!}\int_{0}^{1} x^k (1 - x)^{n-k} \dd x \\
    \end{derive}

    根据分部积分法,有

    \begin{equation}
        \int_{0}^{1} x^k (1 - x)^{n-k} \dd x = \frac{n-k}{k+1} x^{k+1} (1-x)^{n-k-1}
    \end{equation}

    又已知$k=n$时,有$\int_{0}^{1} x^{n} \dd x = \frac{1}{n+1}$,从而

    \begin{derive}[\int_{0}^{1} x^k (1 - x)^{n-k} \dd x]
        &= \frac{n-k}{k+1} x^{k+1} (1-x)^{n-k-1} \\
        &= \frac{n-k}{k+1}\frac{n-k-1}{k+1} x^{k+2} (1-x)^{n-k-2} \\
        &\vdots \\
        &= \frac{n-k}{k+1}\frac{n-k-2}{k+2}\dots\frac{1}{n}\int_{0}^{1} (1-x)^{n}\dd x\\
        &= \frac{n-k}{k+1}\frac{n-k-2}{k+2}\dots\frac{1}{n}\frac{1}{n+1}\\
        &= \frac{k!(n-k)!}{(n+1)!}
    \end{derive}

    因此

    \begin{derive}[\cexpt{S_k}{N(t) = n}]
        &= \frac{n!t}{(k-1)!(n-k)!}\int_{0}^{1} x^k (1 - x)^{n-k} \dd x \\
        &= \frac{n!t}{(k-1)!(n-k)!}\frac{k!(n-k)!}{(n+1)!} \\
        &= \frac{kt}{n+1}
    \end{derive}

    \paragraph*{2.9} 根据\textbf{2.5},有

    \begin{equation}
        f(t_k|N(t) = n)_{t_i, i\not = k} = \frac{n! t_k^{k-1} (t - t_k)^{n-k}}{(k-1)!(n-k)!t^n}
    \end{equation}

    \paragraph*{2.10} 由题,泊松过程的参数$\lambda = 1/2$。首先计算$S_n$的条件概率密度

    \begin{derive}[f_{S_n}(t)]
        &=\frac{\dd F_{S_n}(t)}{\dd t} \\
        &=\frac{\dd \left(\sum_{j=n}^\infty \frac{(\lambda t)^j}{j!}e^{-\lambda t}\right)}{\dd t} \\
        &= \sum_{j=n}^\infty \left(\lambda \frac{(\lambda t)^{j-1}}{(j-1)!}e^{-\lambda t} - \lambda \frac{(\lambda t)^j}{j!} e^{-\lambda t}\right) \\
        &= \lambda e^{-\lambda t}\frac{(\lambda t)^{n-1}}{(n-1)!}
    \end{derive}

    因此,可以计算$P(S_n\geq x)$:

    \begin{derive}[P(S_n\geq x)]
        &= \int_{x}^\infty f_{S_n}(t)\dd t \\
        &= \int_{x}^\infty \lambda e^{-\lambda t}\frac{(\lambda t)^{n-1}}{(n-1)!} \dd t \\
        &= \frac{\lambda^n}{(n-1)!}\int_{x}^\infty t^{n-1}e^{-\lambda t}\dd t
    \end{derive}

    将$\lambda = 1/2$代入,得:

    \begin{equation}
        P(S_n\geq x) = \frac{1}{2^n(n-1)!}\int_x^\infty t^{n-1}e^{-t/2}\dd t
    \end{equation}

    \paragraph*{2.26}

    \begin{subquestions}
        \item[\textbf{(1)}] 设充分小的$\delta = O(t_5 - t_2)$,则:

        \begin{derive}[f(t_2, t_5)]
            &= \begin{cases}
                \lim_{\delta\rightarrow 0}\frac{1}{\delta ^ 2}\prob{t_2 - \frac \delta 2\leq S_2 < t_2 + \frac \delta 2, t_5 - \frac \delta 2\leq S_5 < t_5 + \frac \delta 2} & 0 < t_2 < t_5 \\
                0 & \text{otherwise}
            \end{cases}
        \end{derive}

        设$a, b, c, d$分别为$t_2 - \frac \delta 2, t_2 + \frac \delta 2, t_5 - \frac \delta 2, t_5 + \frac \delta 2$,则$\prob{a\leq S_2 < b, c\leq S_5 < d}$可以按照如下方式计算

        \begin{derive}[&\prob{a\leq S_2 < b, c\leq S_5 < d}\\]
            =& \prob{N(a) = 1, N(b) - N(a) = 1, N(c) - N(b) = 2, N(d) - N(c) = 1} \\
            =& \prob{N(a) = 1}\prob{N(b) - N(a) = 1}\prob{N(c) - N(b) = 2}\prob{N(d) - N(c) = 1} \\
            =& \left(\lambda t_2 e^{-\lambda t_2}\right)\left(\lambda \delta e^{-\lambda \delta}\right)\left(\frac{(\lambda (t_5 - t_2))^2}{2}e^{-\lambda (t_5 - t_2)}\right)\left(\lambda \delta e^{-\lambda \delta}\right) \\
            =& \frac{\lambda ^5 t_2 (t_5 - t_2)^2 \delta^2}{2} e^{-\lambda (t_5 + 2\delta)}
        \end{derive}

        因此,有

        \begin{derive}[&\lim_{\delta\rightarrow 0}\frac{1}{\delta ^ 2}\prob{t_2 - \frac \delta 2\leq S_2 < t_2 + \frac \delta 2, t_5 - \frac \delta 2\leq S_5 < t_5 + \frac \delta 2}\\]
            =& \lim_{\delta\rightarrow 0}\frac{\lambda ^5 t_2 (t_5 - t_2)^2 \delta^2}{2\delta ^2} e^{-\lambda (t_5 + 2\delta)} \\
            =& \frac{\lambda ^5 t_2 (t_5 - t_2)^2}{2} e^{-\lambda (t_5)}
        \end{derive}

        即

        \begin{equation}
            f(t_2, t_5) = \begin{cases}
                \frac{\lambda ^5 t_2 (t_5 - t_2)^2}{2} e^{-\lambda (t_5)} & 0 < t_2 < t_5 \\
                0 & \text{otherwise}
            \end{cases}
        \end{equation}

        \item[\textbf{(3)}] 给定$N(t) = 1$,则

        \begin{enumerate}
            \item $S_1$服从区间$[0, t]$上的均匀分布。
            \item $S_2$即$N(t) = 1$时的$W(t) + t$,且$\prob{W(t) \leq x} = 1 - e^{-\lambda x}, \forall x\geq 0$。
        \end{enumerate}

        由于$S_2 = W(t) + t$与$S_1$独立,则

        \begin{derive}[f(t_1, t_2)]
            &= f(t_1)f(t_2) \\
            &= \frac{1}{t} \left(\lambda e^{-\lambda (t_2 - t)}\right) \\
            &= \frac{\lambda}{t} e^{-\lambda (t_2 - t)}
        \end{derive}
    \end{subquestions}

    \paragraph*{2.27 (1)} 由题

    \begin{enumerate}[label=\textit{(\arabic*)}]
        \item $N(0) = N_1(0) + N_2(0) = 0$
        \item $N(t_1), N_(t_2) - N(t_1), \dots, N(t_n) - N(t_{n-1})$相互独立;
        \item $\forall s, t\geq 0, N(s+t) - N(s)\sim P((\lambda_1 + \lambda_2)t)$
    \end{enumerate}

    \textit{(1)}、\textit{(2)}显然,以下证明\textit{(3)}:

    已知$N_1, N_2$分别为参数$\lambda_1, \lambda_2$的时齐泊松过程,则有

    \begin{equation}
        \begin{aligned}
            N_1(s+t) - N_1(s) &\sim P(\lambda_1t) \\
            N_2(s+t) - N_2(s) &\sim P(\lambda_2t)
        \end{aligned}
    \end{equation}

    因此只需证明相互独立的$X_1, X_2$满足$X_1\sim P(\lambda_1t), X_2\sim P(\lambda_2t)\Rightarrow X_1+X_2\sim P((\lambda_1 + \lambda_2)t)$。已知泊松函数的概率分布为$f(x; \lambda) = \frac{e^{-\lambda}\lambda^x}{x!}$,则

    \begin{derive}[\prob{X_1 + X_2 = x}]
        &= \sum_{x_1=0}^x \prob{X_1 = x_1, X_2 = x - x_1} \\
        &= \sum_{x_1=0}^x \left(\prob{X_1 = x_1}\prob{X_2 = x - x_1}\right) \\
        &= \sum_{x_1=0}^x \left(\frac{e^{-\lambda_1}\lambda_1^{x_1}}{x_1!}\frac{e^{-\lambda_2}\lambda_2^{x - x_1}}{(x - x_1)!}\right) \\
        &= \frac{e^{-(\lambda_1 + \lambda_2)}}{x!}\sum_{x_1=0}^x \left[\left(\frac{x!}{x_1!(x-x_1)!}\right)\lambda_1^{x_1}\lambda_2^{x-x_1}\right] \\
        &= \frac{(\lambda_1 + \lambda_2)^x e^{-(\lambda_1 + \lambda_2)}}{x!}
    \end{derive}

    因此$[N_1(s+t) - N_1(s)] + [N_2(s+t) - N_2(s)] = N(s+t) - N(s) \sim P((\lambda_1 + \lambda_2)t)$,\textit{(3)}得证。

    \paragraph*{2.28} 由题,设$X_t$的矩母函数为$\phi_X(u; t)$,$\rho_i$的矩母函数为$\phi_\rho(u)$,则:

    \begin{derive}[\phi_X(u; t)]
        &= E\left(e^{uX(t)}\right) \\
        &= \sum_{n=0}^\infty \prob{N(t) = n}E\left[e^{(uX(t))} |N(t) = n\right] \\
        &= \sum_{n=0}^\infty \frac{(\lambda t)^n}{n!}e^{-\lambda t} E\left[e^{u(\rho_1 + \rho_2 + \cdots + \rho_n)} | N(t) = n\right] \\
        &= \sum_{n=0}^\infty \frac{(\lambda t)^n}{n!}e^{-\lambda t} E\left[e^{u(\rho_1 + \rho_2 + \cdots + \rho_n)}\right] \\
        &= \sum_{n=0}^\infty \frac{(\lambda t)^n}{n!}e^{-\lambda t} \phi_\rho^n (u) \\
        &= e^{\lambda t(\phi_\rho (u) - 1)}
    \end{derive}

    计算$\phi'_X(0; t)$,得

    \begin{equation}
        \phi'_X(0; t) = \lambda t\phi'_\rho(0) = \lambda tE\rho_1
    \end{equation}

    因此$EX_t = \lambda tE\rho_1 = \lambda t$

    式中,$\rho_1$的矩母函数为

    \begin{derive}[\phi_{\rho}(u)]
        &= Ee^{u\rho} \\
        &= \sum_{k=1}^\infty e^{uk} (1-p)^{k-1}p \\
        &= -\frac{(p-1) p e^u}{(1-p) \left(p e^u-e^u+1\right)} 
    \end{derive}
\end{document}