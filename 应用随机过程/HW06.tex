\documentclass{../notes}

\title{应用随机过程 HW06}

\begin{document}
    \maketitle

    \paragraph*{2.5} 已知$S_1, S_2, \dots, S_n$的条件概率密度为

    \begin{equation}
        f(t_1, t_2, \dots, t_n | N(t) = n) = \begin{cases}
            \frac{n!}{t^n} & 0 < t_1 < t_2 < \cdots < t_n \leq t \\
            0 & \text{otherwise} \\
        \end{cases}
    \end{equation}

    因此,对某个给定的$k\leq n$,有

    \begin{derive}[f(t_k|N(t) = n)_{t_i, i\not = k}]
        &= \int_0^{t_2}\cdots\int_0^{t_k} \int_{t_k} ^t \cdots \int_{t_{n-1}} ^t \frac{n!}{t^n} \dd t_n \cdots \dd t_{k+1} \dd t_{k-1} \cdots \dd t_1 \\
        &= \int_0^{t_2}\cdots\int_0^{t_k} \int_{t_k} ^t \cdots \int_{t_{n-2}} ^t \frac{n!(t - t_{n-1})}{t^n} \dd t_{n-1} \cdots \dd t_{k+1} \dd t_{k-1} \cdots \dd t_1 \\
        &= \int_0^{t_2}\cdots\int_0^{t_k} \int_{t_k} ^t \cdots \int_{t_{n-3}} ^t \frac{n!(t - t_{n-2})^2}{2!\times t^n} \dd t_{n-2} \cdots \dd t_{k+1} \dd t_{k-1} \cdots \dd t_1 \\
        &\vdots \\
        &= \frac{n!(t-t_k)^{n-k}}{(n-k)!t^n} \int_{0}^{t_2}\cdots \int_0^{t_k} \dd t_{k-1} \cdots \dd t_1 \\
        &= \frac{n! t_k^k (t - t_k)^{n-k}}{(n-k)!t^n} \int_{0}^{t_2}\cdots \int_0^{t_{k-1}} t_{k-2} \dd t_{k-2} \cdots \dd t_1 \\
        &\vdots \\
        &= \frac{n! t_k^{k-1} (t - t_k)^{n-k}}{(k-1)!(n-k)!t^n}
    \end{derive}

    从而

    \begin{derive}[\cexpt{S_k}{N(t) = n}]
        &= \int_{0}^{t} t_k f(t_k|N(t) = n)_{t_i, i\not = k} \dd t_{k} \\
        &= \frac{n!}{(k-1)!(n-k)!t^n}\int_{0}^{t} t_k^k (t - t_k)^{n-k} \dd t_{k} \\
        &= \frac{n!t}{(k-1)!(n-k)!}\int_{0}^{1} x^k (1 - x)^{n-k} \dd x \\
    \end{derive}

    根据分部积分法,有

    \begin{equation}
        \int_{0}^{1} x^k (1 - x)^{n-k} \dd x = \frac{n-k}{k+1} x^{k+1} (1-x)^{n-k-1}
    \end{equation}

    又已知$k=n$时,有$\int_{0}^{1} x^{n} \dd x = \frac{1}{n+1}$,从而

    \begin{derive}[\int_{0}^{1} x^k (1 - x)^{n-k} \dd x]
        &= \frac{n-k}{k+1} x^{k+1} (1-x)^{n-k-1} \\
        &= \frac{n-k}{k+1}\frac{n-k-1}{k+1} x^{k+2} (1-x)^{n-k-2} \\
        &\vdots \\
        &= \frac{n-k}{k+1}\frac{n-k-2}{k+2}\dots\frac{1}{n}\int_{0}^{1} (1-x)^{n}\dd x\\
        &= \frac{n-k}{k+1}\frac{n-k-2}{k+2}\dots\frac{1}{n}\frac{1}{n+1}\\
        &= \frac{k!(n-k)!}{(n+1)!}
    \end{derive}

    因此

    \begin{derive}[\cexpt{S_k}{N(t) = n}]
        &= \frac{n!t}{(k-1)!(n-k)!}\int_{0}^{1} x^k (1 - x)^{n-k} \dd x \\
        &= \frac{n!t}{(k-1)!(n-k)!}\frac{k!(n-k)!}{(n+1)!} \\
        &= \frac{kt}{n+1}
    \end{derive}

    \paragraph*{2.9} 根据\textbf{2.5},有

    \begin{equation}
        f(t_k|N(t) = n)_{t_i, i\not = k} = \frac{n! t_k^{k-1} (t - t_k)^{n-k}}{(k-1)!(n-k)!t^n}
    \end{equation}

    \paragraph*{2.10} 由题,泊松过程的参数$\lambda = 1/2$。首先计算$S_n$的条件概率密度

    \begin{derive}[f_{S_n}(t)]
        &=\frac{\dd F_{S_n}(t)}{\dd t} \\
        &=\frac{\dd \left(\sum_{j=n}^\infty \frac{(\lambda t)^j}{j!}e^{-\lambda t}\right)}{\dd t} \\
        &= \sum_{j=n}^\infty \left(\lambda \frac{(\lambda t)^{j-1}}{(j-1)!}e^{-\lambda t} - \lambda \frac{(\lambda t)^j}{j!} e^{-\lambda t}\right) \\
        &= \lambda e^{-\lambda t}\frac{(\lambda t)^{n-1}}{(n-1)!}
    \end{derive}

    因此,可以计算$P(S_n\geq x)$:

    \begin{derive}[P(S_n\geq x)]
        &= \int_{x}^\infty f_{S_n}(t)\dd t \\
        &= \int_{x}^\infty \lambda e^{-\lambda t}\frac{(\lambda t)^{n-1}}{(n-1)!} \dd t \\
        &= \frac{\lambda^n}{(n-1)!}\int_{x}^\infty t^{n-1}e^{-\lambda t}\dd t
    \end{derive}

    将$\lambda = 1/2$代入,得:

    \begin{equation}
        P(S_n\geq x) = \frac{1}{2^n(n-1)!}\int_x^\infty t^{n-1}e^{-t/2}\dd t
    \end{equation}

    \paragraph*{2.26} 已知$S_1, S_2, \cdots S_n$在$N(t) = n$时的联合概率密度为

    \begin{equation*}
        f(t_1, t_2, \cdots, t_n) = \begin{cases}
            \frac{n!}{t^n} & 0 < t_1 < t_2 < \cdots < t_n \leq t \\
            0 & \text{otherwise}
        \end{cases}
    \end{equation*}

    因此
\end{document}