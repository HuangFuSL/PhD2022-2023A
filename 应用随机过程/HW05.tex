\documentclass{../notes}

\title{应用随机过程 HW05}

\begin{document}
    \maketitle

    \paragraph*{3.4} 根据一步转移概率矩阵$\bs P$可知

    \begin{enumerate}
        \item 随机过程$X$非周期
        \item $X$的三个状态相互可达,因此随机过程为不可约马氏链
    \end{enumerate}

    由于$X$不可约,且状态总数有限,因此存在状态$i$,$f_{ii} = 1\land \mu_i < \infty$,从而$X$中各个状态均为正常返。又因$X$非周期,则$X$有唯一的不变分布

    \begin{equation}
        \bs \pi = \left(\pi_i + \frac{1}{\mu_i}\right)_{i\in S}
    \end{equation}

    解方程$\bs \pi \bs P = \bs \pi$,得

    \begin{equation}
        \bs \pi = \begin{bmatrix}
            \frac{14}{29} & \frac{8}{29} & \frac{7}{29}
        \end{bmatrix}
    \end{equation}

    因此状态的平均返回时间为$\bs \mu = \bs 1 / \bs \pi$,其中“$/$”为逐元素的除法,即$\mu_1 = \frac{29}{14}, \mu_2 = \frac{29}{8}, \mu_3 = \frac{29}{7}$。稳定经济状态下,增长状态的概率为$\pi_1 = \frac{14}{29}$,减少状态的概率为$\pi_3 = \frac{7}{29}$。

    \paragraph*{3.5} 解方程$\bs \pi \bs P = \bs \pi$,得

    \begin{equation}
        \bs \pi = \frac{1}{5508}\begin{bmatrix}
            392 & 753 & 1269 & 1838 & 1256
        \end{bmatrix}
    \end{equation}

    因此$\mu_1 = 1 / \pi_1 = \frac{5508}{392} = 14.0$

    \paragraph*{3.6} 当$\sum_{k=0}^\infty ka_k < 1$时,有$E\xi < 1$,即顾客到达速率比服务速率慢。

    首先证明$X$中各个状态均为正常返,即$\mu_i < \infty, \forall i\in S, S = \N$。显然$X$为不可约、非周期的马氏链,因此只需证明$X$中的某个状态正常返即可。此处证明状态$0$为正常返,即$\mu_0 < \infty, f_{00} = 1$。

    设$\xi$的分布列为$\prob{\xi = k} = a_k$,则$X$的状态转移矩阵$\bs P$为:

    \begin{equation}
        \bs P = \begin{bmatrix}
            a_0 & a_1 & a_2 & a_3 & \dots \\
            a_0 & a_1 & a_2 & a_3 & \dots \\
            0 & a_0 & a_1 & a_2 & \dots \\
            0 & 0 & a_0 & a_1 & \dots \\
            \vdots & \vdots & \vdots & \vdots & \ddots
        \end{bmatrix}
    \end{equation}

    即

    \begin{equation}
        p_{ij} = \begin{cases}
            a_{j} & i = 0 \\
            a_{j-i+1} & i > 0\land i - j < 1 \\
            0 & i - j \geq 2
        \end{cases}
    \end{equation}

    $f_{00}^{(1)} = a_0$,根据$f_{00}^{(1)}$推算$f_{00}^{(n)}, n > 1$:

    \begin{equation}
        \begin{aligned}
            f_{i0}^{(2)} &= \sum_{j>0}^{\infty} p_{ij}f_{j0}^{(1)} = 
        \end{aligned}
    \end{equation}

    \paragraph*{3.7}

    \begin{subquestions}
        \item 解方程$\bs \pi \bs P = \bs \pi$,解得

        \begin{equation}
            \bs \pi = \frac{1}{62}\begin{bmatrix}
                21 & 23 & 18
            \end{bmatrix}
        \end{equation}

        由于$X$为非周期、不可约的有限状态马氏链,因此$X$为正常返马氏链,平稳分布$\bs \pi$即为极限分布。对于极限$\lim_{n\rightarrow \infty}\bs P^n$,若该极限存在,则

        \begin{equation}
            \left(\lim_{n\rightarrow \infty}\bs P^n\right)\bs P = \lim_{n\rightarrow \infty}\bs P^n
        \end{equation}

        因此,有

        \begin{equation}
            \lim_{n\rightarrow \infty}\bs P^n = \frac{1}{62}\begin{bmatrix}
                21 & 23 & 18 \\
                21 & 23 & 18 \\
                21 & 23 & 18
            \end{bmatrix}
        \end{equation}
        \item 当初始分布$\bs \pi(0) = \bs\pi$时该马氏链为平稳序列。此时,有

        \begin{eqnarray}
            E(X_n) &= \sum_{i\in S} i\pi_i = \frac{121}{62} \\
            E(X_n^2) &= \sum_{i\in S} i^2\pi_i = \frac{275}{62} \\
        \end{eqnarray}

        方差$D(X_n) = E(X_n^2) - E^2(X_n) = \frac{2409}{3844}$
    \end{subquestions}

    \paragraph*{3.18}

    \begin{subquestions}
        \item 先求$\cprob{T_1=k}{X_0=1}$:

        由于状态$i$只能移动到状态$i-1$或状态$i+1$,且$\cprob{X_{n+1} = i+1}{X_{n} = i} = p$,$\cprob{X_{n+1} = i-1}{X_{n} = i} = q$

        \begin{equation}
            \begin{cases}
                \cprob{T_1=k}{X_0=1} = p^kq^{k+1} & k=2i+1, i\geq 0 \\
                \cprob{T_1=k}{X_0=1} = p^{k+2}q^{k} & k=2i, i\geq 1 \\
            \end{cases}
        \end{equation}

        因此

        \begin{derive}[\cprob{X_{T_1} = 3}{X_0 = 1}]
            &= p^2\sum_{k=0}^\infty p^kq^k \\
            &= \frac{p^2}{1-p+p^2} \\
        \end{derive}

        \item 先求$\cprob{T_2=k}{X_0=2}$,仅考虑$S' = \{0,1,2,3,4,5\}\subset S$,并将$i\leq 0$视为$0$、$i\geq 5$视为$5$,则状态转移矩阵为

        \begin{equation}
            \bs P = \begin{bmatrix}
                q & p & 0 & 0 & 0 & 0 \\
                q & 0 & p & 0 & 0 & 0 \\
                0 & q & 0 & p & 0 & 0 \\
                0 & 0 & q & 0 & p & 0 \\
                0 & 0 & 0 & q & 0 & p \\
                0 & 0 & 0 & 0 & q & p \\
            \end{bmatrix}
        \end{equation}

        初始状态为$i=2$,则$\cprob{T_2=k}{X_0=2} = q\cprob{T_2=k-1}{X_1=1} + p\cprob{T_2=k-1}{X_1=3}$。对于$\cprob{T_2=k-1}{X_1=1}$即$\cprob{T_2=k}{X_0=1}$,有

        \begin{equation}
            \cprob{T_2=k}{X_0=1} = \begin{cases}
                q & k = 1 \\
                p\cprob{T_2=k-1}{X_0=2} & k\geq 2
            \end{cases}
        \end{equation}

        对于$\cprob{T_2=k-1}{X_1=3}$即$\cprob{T_2=k}{X_0=3}$,有

        \begin{equation}
            \cprob{T_2=k}{X_0=3} = q\cprob{T_2=k-1}{X_1=2} + p\cprob{T_2=k-1}{X_1=4}
        \end{equation}

        同理,对于$\cprob{T_2=k-1}{X_1=4}$即$\cprob{T_2=k}{X_0=4}$,有

        \begin{equation}
            \cprob{T_2=k}{X_0=4} = \begin{cases}
                p & k = 1 \\
                q\cprob{T_2=k-1}{X_1=3} & k\geq 2
            \end{cases}
        \end{equation}

        设$a_k = \cprob{T_2=k}{X_0=2}, b_k = \cprob{T_2=k}{X_0=3}$,则$\forall k > 2$,有

        \begin{equation}
            \begin{aligned}
                a_k &= qpa_{k-2} + pb_{k-1} \\
                b_k &= qa_{k-1} + qpb_{k-2} \\
            \end{aligned}
        \end{equation}

        且数列首项$a_1=b_1=0, a_2=q^2, b_2=p^2$,联立解得

        \begin{equation}
            a_{k+4} - 3pqa_{k+2} + p^2q^2a_{k} = 0
        \end{equation}

        设$F_i, i\geq 1$为Fibonacci数列,则有

        \begin{equation}
            F_k = (pq)^{-n/2}a_k
        \end{equation}

        因此,$\cprob{T_2=k}{X_0=2}=a_k$即

        \begin{equation}
            a_k = \begin{cases}
                F_{k-1}(pq)^{k/2}p^{3/2}q^{-3/2} & k = 2n + 1, n\in \N \\
                F_{k-1}(pq)^{k/2}q/p & k = 2n, n\in \N_+
            \end{cases}
        \end{equation}

        \begin{derive}[\cprob{X_{T_2} = 5}{X_0=2}]
            &= \sum_{n=1}^{\infty} a_{2n-1} \\
            &= \frac{p^3}{p^2q^2 - 3pq + 1}
        \end{derive}

    \end{subquestions}

    \paragraph*{3.20 (1)} 由题,状态转移矩阵$\bs P$为

    \begin{equation}
        \bs P = \begin{bmatrix}
            \frac{5}{8} & \frac{1}{4} & \frac{1}{8} \\
            \frac{1}{3} & \frac{1}{2} & \frac{1}{6} \\
            \frac{3}{4} & \frac{1}{4} & 0
        \end{bmatrix}
    \end{equation}

    因此马氏链$X$为非周期、不可约的有限状态马氏链,从而$X$存在平稳分布$\bs \pi$。解方程$\bs \pi \bs P = \bs \pi$,解得

    \begin{equation}
        \bs \pi = \frac{1}{81}\begin{bmatrix}
            44 & 27 & 10
        \end{bmatrix}
    \end{equation}

    由于$X$存在平稳分布,则

    \begin{equation}
        \lim_{n\rightarrow \infty} \bs P = \frac{1}{81}\begin{bmatrix}
            44 & 27 & 10 \\
            44 & 27 & 10 \\
            44 & 27 & 10
        \end{bmatrix}
    \end{equation}

    可知$\cexpt{X_n}{X_0 = 1} = \frac{128}{81}$

    \paragraph*{2.1}

    \begin{subquestions}
        \item $\{N(t) < n\} = \{S_n > t\}$
        \item 当$N$为泊松过程时,关系$\{N(t) \leq n\} = \{S_n \geq t\}$成立,否则不成立
        \item 当$N$为泊松过程时,关系$\{N(t) > n\} = \{S_n < t\}$成立,否则不成立
        \item $\{W(t) > x\} = \{S_{n+1}\geq x + t, S_{n}\leq t\} = \{N(t+x) - N(t) = 0\}$
    \end{subquestions}

    \paragraph*{2.2} 由题$N(t) - N(s)$为泊松过程,且$N(t) - N(s)$与$N(t)$独立,设随机过程为$N'(t)$,则

    \begin{derive}[\cprob{N(s) = k}{N(t) = n}]
        &= \frac{\prob{N(s) = k, N(t) = n}}{N(t) = n} \\
        &= \frac{\prob{N'(t) = n - k}\prob{N(s) = k}}{N(t) = n} \\
        &= \frac{\poisson{\lambda(t-s)}{n-k}\poisson{\lambda s}{k}}{\poisson{\lambda t}{n}} \\
        &= C_n^k \left(\frac{s}{t}\right)^k\left(1 - \frac{s}{t}\right)^{n-k}
    \end{derive}

    \paragraph*{2.3}

    \begin{subquestions}
        \item \begin{derive}[E\{N(t)N(s+t)\}]
            &= E\left(N^2(t) + N(t)\left(N(s+t) - N(t)\right)\right) \\
            &= E\left(N^2(t)\right) + E(N(t))E(N(s+t) - N(t)) \\
            &= \lambda t + \lambda s + \lambda t + \lambda^2 t^2
        \end{derive}

        \item \begin{derive}[\cexpt{N(s+t)}{N(s)}]
            &= \expt{N(s+t) - N(s)} + N(s) \\
            &= \lambda t + N(s)
        \end{derive}

        因此$\prob{\cexpt{N(s+t)}{N(s)} = \lambda t + m} = \prob{N(s) = m} = \frac{(\lambda s)^m}{m!}e^{-\lambda s}$

        \item[(4)] \begin{derive}[\lim_{t\rightarrow s} \prob{N(t) - N(s) > \varepsilon}]
            &= \lim_{t\rightarrow s}\prob{N(t-s) > \varepsilon} \\
            &= \lim_{t'\rightarrow 0^+}\prob{N(t') > \varepsilon} \\
            &= \lim_{t'\rightarrow 0^+}\lambda t' + O(t') \\
            &= 0
        \end{derive}
    \end{subquestions}
\end{document}