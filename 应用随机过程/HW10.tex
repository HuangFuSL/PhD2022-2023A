\documentclass{../notes}

\title{应用随机过程 HW10}

\begin{document}
    \maketitle

    \paragraph*{4.3} 记$T_b\land n = \min\{T_b, n\}$

    \begin{subquestions}
        \item 根据\textbf{4.1},有$U_n= X_n - n(p - q)$是鞅,根据停时定理,有$EU_{T_b\land n} = EU_{0} = 0$,因此$E\left[X_{T_b\land n} - (p - q)\cdot (T_b\land n)\right]$,有

        \begin{equation}
            EX_{T_b\land n} = (p-q)E(T_b\land n)
            \label{eq:4.3.1}
        \end{equation}

        根据$T_b$的定义,有$X_{T_b\land n}\leq b$,即$E(T_b\land n) \leq b/(p - q)$

        根据强大数律,有$X_n/n\xrightarrow{a.s.} p - q > 0$,则$X_n\xrightarrow{a.s.} +\infty$,有$P(T_b < +\infty) = 1$,即$T_b$收敛。当$n\rightarrow +\infty$时,有$ET_b = \lim_{n\rightarrow\infty}E(T_b\land n) \leq b/(p-q)$。

        而$\left|X_{T_b\land n}\right|\leq \left|T_b\land n\right|\leq T_b$,根据控制收敛定理,有$\lim_{n\rightarrow \infty}EX_{T_b\land n} = E\lim_{n\rightarrow \infty}X_{T_b\land n} = EX_{T_b}$,根据$T_b$定义,$X_{T_b} = b$。代入\eqnref{eq:4.3.1},有

        \begin{equation}
            ET_b = \lim_{n\rightarrow\infty}E(T_b\land n) = \frac{\lim_{n\rightarrow\infty }X_{T_b\land n}}{p-q} = \frac{b}{p-q}
        \end{equation}
        \setcounter{enumi}{2}
        \item 根据\textbf{4.1},有$W_n = U_n^2 - n\left[1 - (p - q)^2\right]$是鞅,根据停时定理,有$EW_{T_b\land n} = EW_0 = 0$,将$EW_{T_b\land n}$展开,有

        \begin{derive}[EW_{T_b\land n}]
            &= E\left[(X_{T_b\land n} - (p - q)(T_b\land n))^2 - 4pq\cdot T_b\land n\right] \\
            &= E\left[X^2_{T_b\land n} + (p-q)^2(T_b\land n)^2 - 2(p-q)X_{T_b\land n}\cdot (T_b\land n) - 4pq\cdot (T_b\land n)\right]
        \end{derive}

        从而

        \begin{equation}
            E[X^2_{T_b\land n}] + (p-q)^2E[(T_b\land n)^2] = 2(p-q)E[X_{T_b\land n}\cdot (T_b\land n)] + 4pqE[T_b\land n]
            \label{eq:4.3.2}
        \end{equation}

        根据$T_b$的定义,有$X_{T_b\land n}\leq b\leq T_b\land n\leq n$,则

        \begin{equation}
            \begin{aligned}
                \left|X_{T_b\land n}\cdot (T_b\land n)\right| &\leq \left(T_b\land n\right)^2 \leq n^2 \\
                X^2_{T_b\land n} &\leq \left(T_b\land n\right)^2 \leq n^2
            \end{aligned}
        \end{equation}

        代入\eqnref{eq:4.3.2},有$2(p-q)E[X_{T_b\land n}\cdot (T_b\land n)] + 4pqE[T_b\land n]\leq 2(p-q)bET_b + 4pqET_b$,而由于$E[X^2_{T_b\land n}]\geq 0$,有$E[X^2_{T_b\land n}] + (p-q)^2E[(T_b\land n)^2]\geq (p-q)^2E[(T_b\land n)^2]$,因此$ET_b^2 = \lim_{n\rightarrow \infty}E\left[(T_b\land n)^2\right] < \infty$,从而

        \begin{equation}
            \begin{aligned}
                X_{T_b\land n}^2&\leq (T_b\land n)^2\leq T_b^2 \\
                \left|X_{T_b\land n}\cdot (T_b\land n)\right|&\leq (T_b\land n)^2\leq T_b^2 \\
            \end{aligned}
        \end{equation}

        与\textbf{(a)}同理,有

        \begin{equation}
            \begin{aligned}
                b^2 + (p-q)^2 ET_b^2 &= 2(p-q)bET_b + 4pqET_b \\
                ET_b^2 &= \frac{b^2}{(p-q)^2} + \frac{4pqb}{(p-q)^3}
            \end{aligned}
        \end{equation}

        则

        \begin{equation}
            DT_b = ET_b^2 - E^2T_b = \frac{b[1 - (p-q)^2]}{(p-q)^3}
        \end{equation}
    \end{subquestions}

    \paragraph*{4.7} 记

    \begin{equation}
        \tilde{S_n} = \sum_{k=1}^n \frac{X_k}{n} = \frac{S_n}{n}
    \end{equation}

    考虑$D\left[X_k / n\right] = DX_k / n^2$,由于$EX_n^2 \leq k$且由\textbf{4.6}可知$EX_k = 0$,故由切比雪夫不等式,有

    \begin{equation}
        \lim_{n\rightarrow \infty} P\left(\left|\frac{S_n}{n}\right| > \varepsilon\right) = \lim_{n\rightarrow \infty} P\left(\left|\tilde S_n\right| > \varepsilon\right) \leq \lim_{n\rightarrow \infty}\frac{n}{\varepsilon^2}\cdot \frac{k}{n^2} = 0
    \end{equation}
\end{document}