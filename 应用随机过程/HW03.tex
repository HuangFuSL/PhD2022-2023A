\documentclass{article}
\usepackage[UTF8]{ctex}
\usepackage{amsmath}
\usepackage{amssymb}

\title{应用随机过程 HW03}
\author{皇甫硕龙}

\newcommand{\N}{\mathbb N}
\newcommand{\R}{\mathbb R}
\newcommand{\Z}{\mathbb Z}
\newcommand{\C}{\mathbb C}
\newcommand{\dd}{\mathrm d}
\renewcommand{\geq}{\geqslant}
\renewcommand{\leq}{\leqslant}
\newcommand{\eqnref}[1]{(\ref{#1})}
\newcommand{\prob}[1]{P\left(#1\right)}
\newcommand{\cprob}[2]{P\left(#1|#2\right)}
\newcommand{\bs}{\boldsymbol}

\begin{document}
    \maketitle

    \paragraph*{1.11} 已知$\{B_n, n\geq 1\}$是$\Omega$的一个分解,则有

    \begin{eqnarray}
        \bigcup _{i=1}^\infty B_i &= \Omega \label{eq:1.11-1} \\
        B_i \cap B_j &= \varnothing & \forall i\not = j \label{eq:1.11-2}
    \end{eqnarray}

    根据\eqref{eq:1.11-1},有$A = A\cap \Omega = A\cap \left(\bigcup _{i=1}^\infty B_i\right) = \bigcup _{i=1}^\infty AB_i$,又因\eqref{eq:1.11-2},有

    \begin{equation}
        \prob A = \prob {\bigcup _{i=1}^\infty AB_i} = \sum_{i=1}^\infty \prob{AB_i}
    \end{equation}

    因此,等式右边

    \begin{equation}
        \begin{aligned}
            &\sum_{i=1}^\infty \cprob {B_i}C \cprob A{B_iC} \\
            =&\sum_{i=1}^\infty \frac{\prob{B_iC}}{\prob C} \frac{\prob {AB_iC}}{\prob {B_iC}} \\
            =&\sum_{i=1}^\infty \frac{\prob {AB_iC}}{C} \\
            =&\frac{\prob {AC}}{C} \\
            =&\cprob AC
        \end{aligned}
    \end{equation}

    \paragraph*{1.28}

    $E[E(X|Y)|Y, Z] = E(X|Y)$证明如下:

    \begin{equation}
        \begin{aligned}
            E[E(X|Y)|Y, Z] &= \sum_j \sum_i E[E(X|Y)|Y=y_i, Z=z_j]I_{\{y=y_i, z=z_j\}} \\
            &= \sum_j \sum_i E[E(X|Y)|Y=y_i, Z=z_j]I_{\{y=y_i\}I_{\{z=z_j\}}} \\
            &= \sum_j \sum_i E[E(X|Y)|Y=y_i]I_{\{y=y_i\}}I_{\{z=z_j\}} \\
            &= \sum_i E[E(X|Y)|Y=y_i]I_{\{y=y_i\}} \\
            &= E(X|Y)
        \end{aligned}
    \end{equation}

    $E[E(X|Y, Z)|Y] = E(X|Y)$证明如下:

    \begin{equation}
        \begin{aligned}
            E[E(X|Y, Z)|Y] &= \sum_i E[E(X|Y, Z)|Y=y_i]I_{\{y=y_i\}} \\
            &= \sum_i E[E(X|Y=y_i, Z)]I_{\{y=y_i\}} \\
            &= \sum_j \sum_i E[E(X|Y=y_i, Z=z_j)]I_{\{y=y_i\}}I_{\{z=z_j\}} \\
            &= \sum_i E(X|Y=y_i)I_{\{y=y_i\}} \\
            &= E(X|Y)
        \end{aligned}
    \end{equation}

    \paragraph*{3.1}

    \subparagraph*{(1)} $X_0$经两步转移到$X_2$,又已知$X_0 = 3$,则

    \begin{equation}
        \begin{aligned}
            E(X_2) & = \begin{bmatrix}
                0 & 0 & 1
            \end{bmatrix}\bs{P}_1^2 \begin{bmatrix}
                1 \\ 2 \\ 3
            \end{bmatrix} \\
            &=\begin{bmatrix}
                0 & 0 & 1
            \end{bmatrix}\begin{bmatrix}
                1 & 0 & 0 \\
                \frac 13 & \frac 29 & \frac 49 \\
                \frac 19 & \frac 29 & \frac 23 \\
            \end{bmatrix} \begin{bmatrix}
                1 \\ 2 \\ 3
            \end{bmatrix} \\
            &= \frac{23}{9}
        \end{aligned}
    \end{equation}

    $E(X_2|X_1)$是随机变量,

    \begin{eqnarray}
        E(X_2|X_1) = \frac 73 & x_1 = 2 \\
        E(X_2|X_1) = \frac 83 & x_1 = 3 
    \end{eqnarray}

    $\pi_i(2) = \prob{X_2 = i}$可以按照如下方式计算:

    \begin{equation}
        \begin{aligned}
            \pi_i(2) &= \begin{bmatrix}
                0 & 0 & 1
            \end{bmatrix}\bs{P}_1^2 \\
            &= \begin{bmatrix}
                \frac 19 & \frac 29 & \frac 23
            \end{bmatrix}
        \end{aligned}
    \end{equation}

    $E(X_3|X_2)$是随机变量

    \begin{eqnarray}
        E(X_3|X_2) = 1 & x_2 = 1 \\
        E(X_3|X_2) = \frac 73 & x_2 = 2 \\
        E(X_3|X_2) = \frac 83 & x_2 = 3 
    \end{eqnarray}

    \subparagraph*{(2)} 

    对于$\cprob{T=k}{X_0=3}$,有:

    \begin{equation}
        \begin{aligned}
            \cprob{T=1}{X_0=3} &= 0 \\
            \cprob{T=2}{X_0=3} &= \frac 19 \\
            \cprob{T=3}{X_0=3} &= \frac {2}{27} \\
            \cprob{T\land 4}{X_0 = 3} &= \frac {100}{27}
        \end{aligned}
    \end{equation}

    \subparagraph*{(3)} 

    对于$T_{11}$,有:

    \begin{equation}
        \begin{aligned}
            \prob{T_{11}=1} &= 0 \\
            \prob{T_{11}=2} &= \frac 13 \\
            \prob{T_{11}=3} &= \frac 29 \\
            \cdots \\
            \prob{T_{11}=k} &= \frac{2^{k-2}}{3^{k-1}}
        \end{aligned}
    \end{equation}

    因此$E(T_{11}) = \sum_{k=2}^\infty k\frac{2^{k-2}}{3^{k-1}} = 4$

    \paragraph*{3.2} 解方程$\det (A - \lambda I) = 0$,即

    \begin{equation}
        \begin{vmatrix}
            1 - a - \lambda & a \\
            b & 1 - b - \lambda \\
        \end{vmatrix} = 0
    \end{equation}

    解得$\lambda_1 = 1, \lambda_2 = 1 - a - b$,对应的特征向量为$x_1 = \begin{bmatrix} 1 & 1 \end{bmatrix}^T, x_2 = \begin{bmatrix} a & -b \end{bmatrix}$

    设$k = (1 - a - b)^n$

    \begin{equation}
        \begin{aligned}
            \bs P^n &= D\Lambda^n D^{-1}  \\
            &= \frac{1}{a+b}\begin{bmatrix}
                1 & a \\
                1 & -b \\
            \end{bmatrix}\begin{bmatrix}
                1 & 0 \\
                0 & k
            \end{bmatrix}\begin{bmatrix}
                b & a \\
                1 & -1 \\
            \end{bmatrix} \\
            &= \frac {1}{a+b}\begin{bmatrix}
                b + ak & a - ak \\
                b - bk & a + bk \\
            \end{bmatrix} \\
            &= \frac {1}{a+b}\left(\begin{bmatrix}
                b & a \\
                b & a
            \end{bmatrix} + k\begin{bmatrix}
                a & -a \\s
                -b & b
            \end{bmatrix}\right)
        \end{aligned}
    \end{equation}

    将$k = (1 - a - b)^n$代入即可

    \paragraph*{3.3} 根据$\bs \Phi(z)$的定义,只需证

    \begin{equation}
        (\bs I- z\bs P)\left(\bs I + z\bs P + z^2\bs P^2 \dots \right) = \bs I
    \end{equation}

    将等式左边展开,得

    \begin{equation}
        \begin{aligned}
            \left(\bs I + z\bs P + z^2\bs P^2 + \dots\right) + \left(z\bs P + z^2\bs P^2 + \dots \right) = \bs I
        \end{aligned}
    \end{equation}

    \paragraph*{3.8} \subparagraph*{(1)} $\bs P$的特征值为$\lambda_1 = 1/2, \lambda_2 = 3/4, \lambda_3 = 1$,对应的特征向量为:

    \begin{equation}
        x_1 = \begin{bmatrix}
            1 \\ 0 \\ 0
        \end{bmatrix} \quad
        x_2 = \begin{bmatrix}
            1 \\ 1 \\ 0
        \end{bmatrix} \quad
        x_3 = \begin{bmatrix}
            1 \\ 1 \\ 1
        \end{bmatrix}
    \end{equation}

    从而$\bs P^n$可以按如下方式计算:

    \begin{equation}
        \begin{aligned}
            \bs P^n &= D\Lambda^n D^{-1} \\
            &= \begin{bmatrix}
                1 & 1 & 1 \\
                0 & 1 & 1 \\
                0 & 0 & 1 \\
            \end{bmatrix}\begin{bmatrix}
                (1/2)^n & 0 & 0 \\
                0 & (3/4)^n & 0 \\
                0 & 0 & 1 \\
            \end{bmatrix}\begin{bmatrix}
                1 & -1 & 0 \\
                0 & 1 & -1 \\
                0 & 0 & 1 \\
            \end{bmatrix} \\
            &= \begin{bmatrix}
                (1/2)^n & (3/4)^n - (1/2)^n & 1 - (3/4)^n \\
                0 & (3/4)^n & 1 - (3/4)^n \\
                0 & 0 & 1
            \end{bmatrix}
        \end{aligned}
    \end{equation}

    因此,$\prob{T_{13} = k} = \frac 14 \left(1 - \left(1 - \left(\frac 34\right)^{k-1}\right)\right) = \frac{3^{k-1}}{4^k}$,从而$ET_{13} = \sum_{k=1}^\infty k\prob{T_{13} = k} = 4$

    \subparagraph*{(2)} 由于状态$2$不能反向向状态$1$转变,状态$3$不能反向向状态$2$转变,则:

    \begin{equation}
        \begin{aligned}
            f_{11} &= \frac 12 \\
            f_{22} &= \frac 34 \\
            f_{33} &= 1
        \end{aligned}
    \end{equation}

    \subparagraph*{(3)} 根据\textbf{(1)},当$n\rightarrow \infty$时,有

    \begin{equation}
        \bs P^n = \begin{bmatrix}
            0 & 0 & 1 \\
            0 & 0 & 1 \\
            0 & 0 & 1
        \end{bmatrix}
    \end{equation}

    \paragraph*{3.13} 只需证明

    \begin{equation}
        \cprob{R_{i+1} = a_{i+1}}{R_0=a_0,\dots,R_i=a_i} = \cprob{R_{i+1} = a_{i+1}}{R_i=a_i}
    \end{equation}

    由于$X_i$相互独立,因此$\max\{X_1, \dots X_i\}$与$X_{i+1}$相互独立。因此$R_{i+1} = \max\{a_i, X_{i+1}\}$的分布只与$R_i$的状态有关,而与$R_{0},\dots,R_{i-1}$无关,因此$\{R_i, i\geq 1\}$是马尔可夫链。

    转移概率为

    \begin{equation}
        p_{ij} = \begin{cases}
            \alpha_{j} & j > i \\
            \sum_{k=i}^j \alpha_{k} & j = i \\
            0 & j < i
        \end{cases}
    \end{equation}

    \paragraph*{3.17}

    \begin{equation}
        \begin{aligned}
            &\cprob{X_{n+1}=j}{X_k\in B_k, 0\leq k\leq n-1, X_n=i} \\
            =&\sum_{b_{mk}\in B_k} \cprob{X_{n+1}=j}{X_k\in B_{k}, 0\leq k\leq n-1, X_n = i} \\
            =&\sum_{b_{mk}\in B_k} \frac{\prob {X_{n+1}=j, X_n=i, X_k=b_{mk}, 0\leq k\leq n-1}}{\prob{X_k\in B_{k}, 0\leq k\leq n-1}\prob{X_n=i}} \\
            =&\frac{\prob {X_{n+1}=j, X_n=i}}{\prob{X_n=i}} \sum_{b_{mk}\in B_k} \frac{\prob {X_k=b_{mk}, 0\leq k\leq n-1}}{\prob{X_k\in B_{k}, 0\leq k\leq n-1}} \\
            =&\cprob{X_{n+1}=j}{X_n=i}
        \end{aligned}
    \end{equation}
\end{document}