\documentclass{../notes}

\title{应用随机过程 HW12}

\begin{document}
    \maketitle

    \paragraph*{5.4} 设$B(t)$的概率密度函数为$f(x;t)$,概率分布函数为$F(x;t)$,则$f(x;t) = f(-x;t)$,以下计算$|B(t)|$的概率分布函数,设为$F_1(x;t)$,则$x < 0$时,有$F_1(x;t) = 0$,当$x\geq 0$时,有

    \begin{derive}[F_1(x;t)]
        &= P(|B(t)| \leq x) \\
        &= P(-x \leq B(t)\leq x) \\
        &= F(x) - F(-x) \\
        &= \int_{-x}^x f(u;t) \dd u \\
        &= 2\int_{0}^x f(u;t) \dd u \\
        &= 2\int_{0}^x \frac{1}{\sqrt{2\pi t}}e^{-\frac{u^2}{2t}}\dd u
    \end{derive}

    以下计算$\left|\min\limits_{0\leq s\leq t} B(s)\right|$的概率分布

    \begin{derive}[P\left(\left|\min\limits_{0\leq s\leq t} B(s)\right| \geq x\right)]
        &= P\left(-x\geq \min\limits_{0\leq s\leq t} B(s)\right) \\
        &= P\left(x\leq \max\limits_{0\leq s\leq t} B(s)\right)
    \end{derive}

    因此$\left|\min\limits_{0\leq s\leq t} B(s)\right|$与$\max\limits_{0\leq s\leq t} B(s)$同分布,而

    \begin{equation}
        \begin{aligned}
            & P\left(x\leq \max\limits_{0\leq s\leq t} B(s)\right) \\
            =&P(T_x\leq t) \\
            =&P(T_x\leq t, B(t)\geq x) + P(T_x\leq t, B(t) > x) \\
            =&P(B(t)\geq x) + P(T_x\leq t, B(t) > x) \\
            =& 2P(B(t)\geq x)
        \end{aligned}
    \end{equation}

    从而$P\left(\max\limits_{0\leq s\leq t} B(s)\leq x\right) = 2P(B(t)\leq x) = P(|B(t)|\leq x)$,即$|B(t)|$与$\max\limits_{0\leq s\leq t} B(s)$同分布。

    以下计算$\delta (t) = M(t) - B(t)$的分布

    \begin{derive}[\delta(t)]
        &= \left(\max\limits_{0\leq s\leq t} B(s)\right) - B(t) \\
        &= \max\limits_{0\leq s\leq t} \left[B(s) - B(t)\right] \\
        &= \max\limits_{0\leq s\leq t} -B(t - s) \\
        &= \max\limits_{0\leq s\leq t} B(t - s) \\
        &=\max\limits_{0\leq s\leq t} B(s) \\
    \end{derive}

    因此$\delta(t)$与$\max\limits_{0\leq s\leq t} B(s)$同分布。
    \paragraph*{5.9}

    \begin{derive}[\alpha(x)]
        &= \lim_{h\rightarrow 0}\frac{E[\eta (t+h) - \eta(t)|\eta(t) = x]}{h} \\
        &= \lim_{h\rightarrow 0}\frac{E\left[\eta(t)\left(\eta(h) - 1\right)|\eta(t) = x\right]}{h} \\
        &=x\lim_{h\rightarrow 0}\frac{E\left(\eta(h) - 1\right)}{h} \\
        &=x\lim_{h\rightarrow 0}\frac{e^{\frac{1}{2}h} - 1}{h} = \frac{x}{2}
    \end{derive}

    \begin{derive}[\beta(x)]
        &= \lim_{h\rightarrow 0}\frac{E[[\eta (t+h) - \eta(t)]^2|\eta(t) = x]}{h} \\
        &= x^2\lim_{h\rightarrow 0}\frac{E[(\eta(h)-1)^2]}{h} \\
        &= x^2\lim_{h\rightarrow 0}\frac{E(\eta(h)^2 - 2\eta(h) + 1)}{h} \\
        &= x^2\lim_{h\rightarrow 0}\frac{e^{2h} - 2e^{h/2} + 1}{h} \\
        &= x^2
    \end{derive}
    \paragraph*{5.16}

    根据全概率公式,有

    \begin{equation}
        P(T_1 < T_{-1} < T_2) = P(T_1 < T_{-1}) - P(T_1 < T_{-1}, T_2 < T_{-1})
    \end{equation}

    根据布朗运动的对称性,有$P(T_1 < T_{-1}) = 1/2, P(T_1 < T_{-1}, T_2 < T_{-1}) = P(T_1 < T_2 < T_{-1})=P(T_2 < T_{-1})$。

    根据停时定理,有$P(T_2 < T_{-1}) = 1/3$,因此$P(T_1 < T_{-1} < T_2) = 1/2 - 1/3 = 1/6$
\end{document}