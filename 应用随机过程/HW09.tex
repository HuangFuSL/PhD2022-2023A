\documentclass{../notes}

\title{应用随机过程 HW09}

\loadgeometry{word-moderate}

\begin{document}
    \maketitle
    \paragraph*{4.1}

    \begin{subquestions}
        \item $\{U_n, n\geq 0\}$又可写作

        \begin{equation}
            U_n = \sum_{k=1}^n \left(Y_k - (p - q)\right)
        \end{equation}

        则$E|U_n| = \left|\sum_{k=1}^n \left(Y_k - (p-q)\right)\right| \leq \sum_{k=1}^n |Y_k - (p - q)| \leq n\max\{1 + q - p, p - q\}<\infty$

        \begin{derive}[E(U_{n+1} | Y_0, Y_1, \cdots, Y_n)]
            &= E(U_{n} + \left(Y_{n+1} - (p - q)\right) | Y_0, Y_1, \cdots, Y_n) \\
            &= E(U_n|Y_0, Y_1, \cdots, Y_n) + E(Y_{n+1} - (p - q)|Y_0, Y_1, \cdots, Y_n) \\
            &= E(U_n|Y_0, Y_1, \cdots, Y_n) \\
            &= U_n
        \end{derive}

        $\{V_n, n\geq 0\}$又可写作

        \begin{equation}
            V_n = \prod_{k=1}^n \left(\frac{q}{p}\right)^{Y_k}
        \end{equation}

        由于$E\left(\frac{q}{p}\right)^{Y_k} = q + p = 1$,且$\{Y_n, n\geq 1\}$独立同分布,则有$EV_n = EV_{n-1} = \cdots = EV_0 = 1$。又因

        \begin{derive}[E(V_{n+1} | Y_0, Y_1, \cdots, Y_n)]
            &= E\left(V_{n}\cdot \left(\frac{q}{p}\right)^{Y_n} \middle| Y_0, Y_1, \cdots Y_n\right) \\
            &= E\left(V_{n} \middle| Y_0, Y_1, \cdots Y_n\right) E\left(\left(\frac{q}{p}\right)^{Y_n} \middle| Y_0, Y_1, \cdots Y_n\right) \\
            &= E\left(V_{n} \middle| Y_0, Y_1, \cdots Y_n\right) \\
            &= V_n
        \end{derive}

        $\{W_n, n\geq 0\}$:设$Y'_n = Y_n - (p - q)$,则有$EY'_n = 0$,$EY'^2_n = \sigma^2 = 1 - (p-q)$,则$\{W_n, n\geq 0\}$可以写作

        \begin{equation}
            W_n = \left(\sum_{k=1}^n Y'_k\right)^2 - n\sigma^2
        \end{equation}

        首先证明$E|W_n| < \infty$,有

        \begin{derive}[E|W_n|]
            &= E\left|\left(\sum_{k=1}^n Y'_k\right)^2 - n\sigma^2\right| \\
            &\leq E\left|\left(\sum_{k=1}^n Y'_k\right)^2 \right| +  n\sigma^2 \\
            &= E\left(\sum_{k=1}^n Y'^2_k + \sum_{i\not = j}Y'_iY'_j\right) + n\sigma^2 \\
            &= 2n(1 - p + q)
        \end{derive}

        另一方面:

        \begin{derive}[E(W_{n+1}|Y'_0, Y'_1, \cdots, Y'_n)]
            &= E\left[\left(Y'_{n+1} + \sum_{k=1}^n Y'_k\right)^2 - (n+1)\sigma^2\middle| Y'_0, Y'_1, \cdots, Y'_n\right] \\
            &= E\left[Y'^2_{n+1} + \left(\sum_{k=1}^n Y'_k\right)^2 + Y'_{n+1}\sum_{k=1}^n Y'_k - (n+1)\sigma^2\middle| Y'_0, Y'_1, \cdots, Y'_n\right] \\
            &=EY'^2_{n+1} + 2E(Y_{n+1}|Y_0, Y_1, \cdots Y_n)\sum_{k=1}^n Y_k + W_n - \sigma^2 \\
            &= W_n
        \end{derive}

        因此$\{U_n, n\geq 0\}, \{V_n, n\geq 0\}, \{W_n, n\geq 0\}$是关于$\{Y_n, n\geq 0\}$的鞅

        \item 由于$E(X_n^+) \leq n < \infty$,只需说明$E(X_{n+1}|Y_0, \cdots, Y_n)\geq X_n$即可,证明如下:

        \begin{derive}[E(X_{n+1}|Y_0, \cdots, Y_n)]
            &= E(X_{n} + Y_{n+1}|Y_0, \cdots, Y_n) \\
            &= E(X_{n}|Y_0, \cdots, Y_n) + E(Y_{n+1}|Y_0, \cdots, Y_n) \\
            &= X_n + p - q
        \end{derive}

        因此$\{X_n, n\geq 0\}$是关于$\{Y_n, n\geq 0\}$的下鞅。

        \item 设相关系数为$\rho$,则$\rho = \Cov(U_{m}, U_{m+n}) / (\sigma_{U_{m}}\sigma_{U_{m+n}})$

        首先计算$U_{m}, U_{m+n}$的协方差,有

        \begin{derive}[\Cov(U_m, U_{m+n})]
            &= E\left[U_mU_{m+n}\right] - EU_mEU_{m+n} \\
            &= E\left[U_mU_{m+n}\right] \\
            &= E\left[X_mX_{m+n}\right] - m(p-q)EX_{m+n} - (m+n)(p-q)EX_m + m(m+n)(p-q)^2 \\
            &= E\left[X_mX_{m+n}\right] - m(m+n)(p-q)^2 \\
            &= EX_m^2 + EX_m E\left(\sum_{k=1}^n Y_{m+k}\right) - m(m+n)(p-q)^2 \\
            &= EX_m^2 - m^2(p-q)^2 \\
            &= E\left(\sum_{k=1}^m Y_k^2\right) + E\left(\sum_{i\not = j}\left(Y_iY_j\right)\right) - m^2(p-q)^2 \\
            &= m(1 - (p - q)^2) \\
            &= 4mpq
        \end{derive}

        其次计算$\sigma_{U_{m}}^2$,有

        \begin{derive}[\sigma_{U_{m}}^2]
            &= EU_m^2 \\
            &= E\left(\sum_{k=1}^m \left(Y_k - p + q\right)^2\right) + E\left(\sum_{i\not = j}\left((Y_i - p + q)(Y_j - p + q)\right)\right) \\
            &= 4mpq
        \end{derive}

        则相关系数为

        \begin{derive}[\rho]
            &= \frac{\Cov(U_{m}, U_{m+n})}{\sigma_{U_{m}}\sigma_{U_{m+n}}} \\
            &= \sqrt{\frac{m}{m+n}}
        \end{derive}
    \end{subquestions}

    \paragraph*{4.6} 由题$\xi _i$可以写为

    \begin{equation}
        \xi_i = \begin{cases}
            X_i - X_{i-1} & i > 0 \\
            X_0 & i = 0
        \end{cases}
    \end{equation}

    当$i = 0, j\not = 0$时,有

    \begin{derive}[E(\xi_i\xi_j)]
        &= E(X_0(X_j - X_{j-1})) \\
        &= E(E(X_0(X_j - X_{j-1}) | X_0, \cdots X_{j-1})) \\
        &= E(X_0(E(X_j | X_0, \cdots, X_{j-1}) - X_{j-1})) \\
        &= 0
    \end{derive}

    当$ij\not = 0$时,不妨设$i < j$,则

    \begin{derive}[E(\xi_i\xi_j)]
        &= E((X_i - X_{i-1})(X_j - X_{j-1})) \\
        &= E(E((X_i - X_{i-1})(X_j - X_{j-1}) | X_0, \cdots, X_{j-1})) \\
        &= E((X_i - X_{i-1})(E(X_j|X_0, \cdots, X_{j-1}) - X_{j-1})) \\
        &= 0
    \end{derive}

    \paragraph*{4.10 (1)} 已知$P(X_{n} = j | X_{n-1} = i) = \frac{i^j}{j!}e^{-i}$。则

    \begin{derive}[E(X_{n+1} | X_0, X_1, \cdots, X_n)]
        &= E(X_{n+1} | X_n) \\
        &= \sum_{j=1}^\infty jP(X_{n+1} = j | X_n = i) \\
        &= \sum_{j=1}^\infty \frac{i^j}{(j-1)!}e^{-i} \\
        &= i
    \end{derive}

    因此$E(X_{n+1} | X_0, X_1, \cdots, X_n) = X_n$,因此$\{X_n, n\geq 0\}$是鞅。

    \paragraph*{4.23}

    \begin{subquestions}
        \item $\{X_n\lor c, n\geq 0\} = \{\max(X_n, c), n\geq 0\}$,则
        
        \begin{derive}[E(X_{n+1}^+ | Y_0, \cdots, Y_n)]
            &\geq E(X_{n+1} | Y_0, \cdots, Y_n) \\
            &= X_n
        \end{derive}

        \item 令$c = 0$,可知$\{X_n^+, n\geq 0\}$为下鞅。
    \end{subquestions}
\end{document}