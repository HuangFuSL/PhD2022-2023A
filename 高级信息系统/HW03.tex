\documentclass{../notes}

\title{基于均匀性与冗余性的结构覆盖测度的改良 \\ {\Large 《高级信息系统》小组作业报告}}
\author{姚欣培\;王元翔\;钱思涵\;皇甫硕龙}

\newcommand{\Cov}{\text{Cov}}
\newcommand{\Sim}{\text{Sim}}

\begin{document}
    \maketitle
    \begin{abstract}
        本文简要介绍与分析了多样性信息子集提取问题中,基于信息熵的信息结构覆盖测度的基本内容。随后从不同的角度提出了几种新的结构覆盖测度,从数据离散程度的角度提出了基于极差与方差的结构覆盖测度,从势的角度提出了由冗余替代相似度的信息熵计算,对新测度的相关性质予以证明,并通过简单的算例证明其有效性。
    \end{abstract}

    \section*{前言}

    Ma等人\cite{ma2017content}针对多样性信息子集的提取问题,利用信息熵作为总信息覆盖度量的一部分来模拟信息结构覆盖。本篇报告将在第一部分对Ma等人的研究及提出的结构覆盖测度进行简要的介绍与分析,在第二部分介绍与新的结构覆盖测度相关的理论背景知识,第三部分尝试从不同的角度提出几种新的结构覆盖测度,对相关性质进行讨论与证明,并举出计算的实例。

    \section{基于信息熵的结构覆盖测度}

    \subsection*{背景知识}

    \begin{definition}
        离散程度:通过随机地观测变量各个取值之间的差异程度,数据的离散程度即衡量一组数据的分散程度如何。
    \end{definition}

    \begin{definition}
        信息熵:假定当前样本集合D中第k类样本所占的比例为$p_k$
        \begin{equation}
            \text{Information\ Entropy}=-\sum{p_k\cdot\log_2{p_k}}
        \end{equation}

        在一个样本集合中,信息熵反应了样本的离散程度、随机程度。信息熵越大,离散程度越高,当所有样本都是等概率时,随机程度最大为$1$。信息熵越小,离散程度越低,当只有某一个样本概率$100\%$,其它样本概率为$0\%$时,随机程度最小为$0$。
    \end{definition}

    \begin{definition}
        极差:

        \begin{equation}
            \text{Range}=\max{\left(d_i\right)}-\min(d_i)
        \end{equation}

        极差又称范围误差或全距(Range),以$R$表示,是用来表示统计资料中的变异量数(measures of variation),其最大值与最小值之间的差距,即最大值减最小值后所得之数据。它反应了数据样本的数值范围,是最基本的衡量数据离散程度的方式,受极值影响较大。
    \end{definition}

    \begin{definition}
        方差:

        \begin{equation}
            \text{Variance}=\frac{\sum_{i=1}^{n}\left(d_i-\bar{d}\right)^2}{n}
        \end{equation}

        方差是在概率论和统计方差衡量随机变量或一组数据时离散程度的度量。概率论中方差用来度量随机变量和其数学期望(即均值)之间的偏离程度。统计中的方差(样本方差)是每个样本值与全体样本值的平均数之差的平方值的平均数。
    \end{definition}

    \subsection{文献简介}

    Ma等人从信息内容和结构相结合的角度,研究了在考虑信息覆盖率度量时如何构建一种子集提取方法,以获得多样化的结果集。具体地,在CovC-Select贪婪子模思想的基础上,应用模拟退火法的策略,提出了一种启发式算法CovC+S-Select。在此基础上,进一步提出了一种快速逼近方法FastCovC+S-Select,旨在以有效、高效和稳健的方式提取出不同的子集,并通过评估实验证明了该方法的有效性。此外,文章从信息覆盖、外部标记和人类评价三个角度对11种主要的多样性提取方法与FastCovC+S-Select进行了全面系统的比较研究,并通过对比实验进一步证明了该方法的优越性。

    其中,文章在考虑信息覆盖率时所提出的测度,值得我们进一步的探索与研究,作者们在文章中给定了含$n$个元素的原始集合$D$,以及从$D$中提取出来的规模为$k$的小集合$D'$,为了反映$D'$对于$D$的信息覆盖程度,利用信息熵来建模信息结构覆盖,作为总信息覆盖度量的一部分。具体地,$D'$中的的每一个元素$d_j', j=1,2,\cdots,k$可以被看作是一个隐式的类别标签,从而产生了$k$个天然的类。进而,对于$D$中的每一个元素$d$,可以根据其与$D'$中的的每一个元素$d_j'$的相似度高低,确定其类别$m$,即$m={\arg\max}_{j=1,2,\ldots,k}\left(\Sim\left(d_j^\prime,d\right)\right), m=1,2,\cdots k$。从而$D$中的$n$个元素可以被划分到$k$个子类当中,记为$D_1, D_2, \cdots D_k$。从而,可以得到每个子类$D_j$的信息负载量$n_j^v$,即以$d_j'$为类标签的类别$D_j$中元素的隶属度的和,形成了$d_j'$与$D_j$的对应$n_j^v=\sum_{d\in D_j} \Sim\left(d_j^\prime,d\right)$,从而进一步得到原始集合$D$的信息负载量$n^v=\sum_{j=1}^{k}n_j^v$,作为$D'$与$D$的对应,从而可以基于信息熵,计算$D'$对于$D$的结构覆盖程度$\Cov_s(D^\prime,D)$

    \begin{equation}
        \Cov (D', D) = \begin{cases}
            1 & k=1 \\
            -\frac{1}{\log_2 k}\sum_{j=1}^k \frac{n_j^v}{n^v}\cdot \log_2\left(\frac{n_j^v}{n^v}\right) & k > 1
        \end{cases}
        \label{eq:1}
    \end{equation}

    $\Cov_s(D^\prime,D)$具备一些有用的性质,首先它的取值范围为$(0,1]$,且具有自反性。其次,如果$D$中的信息负载量能够以等价分布传递到$D'$中,则$D'$能够保持最佳的信息结构。第三,如果$D$的信息负载量可以以更近似的分布分配到每个$D_j$上,则$\Cov_s(D^\prime,D)$的取值将更接近于$1$,这对于设计更好的策略来提取具有更高覆盖率的子集具有重要的意义。

    \subsection{分析与思考}

    为了便于后续提出新的结构覆盖测度,我们想要对Ma等人提出的基于信息熵的结构覆盖测度进行进一步的分析。具体地,我们思考各个子类与原始集的信息负载量之比满足什么条件时,$\Cov_s(D^\prime,D)$能够取得最大值,以及此时反映了$D'$是怎样的信息结构。

    首先,证明当信息负载量之比$\frac{n_j^v}{n^v}$服从均匀分布时,$\Cov_s(D^\prime,D)$取得最大值:

    \begin{lemma}
        设随机变量$X$取自样本空间$S = \{X_1, X_2, \cdots, X_n\}$,概率分布为$P(X = X_i) = p_i, i=1, 2, \cdots, n$,且有$\sum_{i=1}^n p_i = 1$。当且仅当$p_1 = p_2 = \cdots = p_n = \frac{1}{n}$时,$-\sum_{i=1}^n p_{i}\log p_i$取最大值。
    \end{lemma}

    \begin{proof}
        考虑函数$g(x) = x\ln x$,有

        \begin{equation}
            \frac{\dd^2 g}{\dd x^2} = \frac{1}{x} > 0
        \end{equation}

        由此,$g(x)$为凸函数。对$g(x)$应用Jensen不等式,有$E(g(x)) \geq g(E(x))$,即

        \begin{equation}
            \frac{\sum_{i=1}^n p_i\log p_i}{n}\geq \frac{\sum_{i=1}^n p_i}{n}\log \frac{\sum_{i=1}^n p_i}{n}
            \label{eq:2}
        \end{equation}

        当且仅当$p_1 = p_2 = \cdots = p_n$时,式\eqnref{eq:2}取等。由\eqnref{eq:2}可得$\sum_{i=1}^n p_i\log p_i\geq \log \frac{1}{n}$,即

        \begin{equation}
            -\sum_{i=1}^n p_i\log p_i \leq \log n
            \label{eq:3}
        \end{equation}

        当且仅当$p_1 = p_2 = \cdots = p_n$时,式\eqnref{eq:3}取等。
    \end{proof}

    接下来,我们考虑信息负载量之比为均匀分布时,$D'$的信息结构是怎样的。基于信息负载量的定义及上述证明,我们不难想到,$\Cov_s(D^\prime,D)$的值越小,说明$D$的纯度越高,某个子类在$D$中占很高的比重,即$D'$中的元素作为分类标签的分类效果并不好,没有体现出$D$本身的多样性,极限情况下,考虑$D$中元素完全属于同一类,说明$D'$完全没有分类效果,$\Cov_s(D^\prime,D)$的值为$0$。当$D$中的元素均匀的分布在不同的子类中时,$\Cov_s(D^\prime,D)$取到最大值$1$,此时说明$D'$ 中的元素很好地反映了原始集的信息结构,分类的效果是显著的。一个便于理解的典型的例子是分层抽样,抽样时样本不同种类个体的比例越接近于总体中不同种类个体的比例,抽样效果往往越好,样本对总体的代表性越好。

    综上,信息熵可以反映出样本的离散程度,即$\Cov_s(D^\prime,D)$可以反映出$\frac{n_j^v}{n^v}$的离散程度,$\Cov_s(D^\prime,D)$越大,$\frac{n_j^v}{n^v}$的离散程度越高,$D'$对$D$的信息结构的覆盖效果越好,当$\frac{n_j^v}{n^v}$取为均匀(相等)的时候,$\Cov_s(D^\prime,D)$最大等于$1$,此时$D'$对$D'$的信息结构有最优的覆盖效果;$\Cov_s(D^\prime,D)$越小,$\frac{n_j^v}{n^v}$的离散程度越小,$D'$对$D$的信息结构覆盖效果越差。

    \section{理论架构}

    \subsection{结构性度量}

    Ma等人提出的结构覆盖度指标首先将大集合$D$通过小集合$D'$的每一个元素$d_i$分成$k$个分割的子集作为小集合在大集合上映射的类别结构,再计算出每一个类别结构的势并进行归一化$\frac{n_j^v}{n^v}$,再计算这些归一化后的类别结构的势的信息熵作为该小集合$D'$在大集合$D$上的结构覆盖度。

    % TODO:重新润色一下

    而在原文计算过程中第二步中根据类别结构的势计算出结构覆盖度的过程中,在原论文中使用信息熵来度量测度结构覆盖度,而信息熵本来也是一种离散程度的度量方式,所以我们想探索使用其他的离散程度度量方法\cite{刘颖1997关于有限集点分布均匀性的度量方法}来衡量给定类别结构的势的情况下小集合对大集合的结构覆盖度。并且我们会证明这些其他的度量方法,也具有良好的取值范围,自反性等特点。

    \paragraph*{极差}

    使用极差指标定义的的结构性度量如下:

    \begin{equation}
        \Cov_s^{\text{range}}(D',D)=1-\left[\max\left(\frac{n_j^v}{n^v}\right)-\min\left(\frac{n_j^v}{n^v}\right)\right]
    \end{equation}

    可以证明,$\Cov_s^{\text{range}}(D',D)$满足如下性质:

    \begin{enumerate}
        \item \textbf{自反性:} 当$D' = D$时,对于任意的$j$,都有$\frac{n_j^v}{n^v}=\frac{1}{k}$,所以$\max{\left(\frac{n_j^v}{n^v}\right)}=\min(\frac{n_j^v}{n^v})=\frac{1}{k}$,所以$\Cov_s^{range}\left(D,D\right)=1$,满足自反性。
        \item \textbf{取值范围:} 由$\frac{n_j^v}{n^v}\in(0,1]$,所以$\max{\left(\frac{n_j^v}{n^v}\right)}-\min{\left(\frac{n_j^v}{n^v}\right)}\in[0,1)$,所以$\Cov_s(D',D)=1-\left[\max\left(\frac{n_j^v}{n^v}\right)-\min\left(\frac{n_j^v}{n^v}\right)\right]\in (0,1]$
        \item \textbf{极值敏感性:} 由于极值会直接影响势的最大值和最小值,从而直接影响到极差统计量的大小,进而影响小集合对大集合对结构覆盖度的指标。
    \end{enumerate}

    \paragraph*{方差}

    使用方差指标定义的结构性度量如下:

    \begin{equation}
        {\Cov_s^{\text{var}}\left(D^\prime,D\right)=1-\frac{\sum_{j=1}^{k}\left(\frac{n_j^v}{n^v}-\frac{1}{k}\right)}{k}}^2
    \end{equation}

    可以证明,$\Cov_s^{\text{var}}\left(D^\prime,D\right)$满足如下性质:

    \begin{enumerate}
        \item \textbf{自反性:} 当$D' = D$时,对于任意的$j$,都有$\frac{n_j^v}{n^v}=\frac{1}{k}$,所以$\left(\frac{n_j^v}{n^v}-\frac{1}{k}\right)^2=0$,所以${1-\frac{\sum_{j=1}^{k}\left(\frac{n_j^v}{n^v}-\frac{1}{k}\right)}{k}}^2=0$,所以$\Cov_s^{\text{var}}\left(D^\prime,D\right)=1$,满足自反性。
        \item \textbf{取值范围:} 因为方差$Var\left(X\right)=E(X-E\left(X\right))^2$,对于$X=\frac{n_j^v}{n^v}$来说,$X\in\left(0,1\right]$,$E\left(X\right)=\frac{1}{k}\in(0,1]$,所以$\left(X-E\left(X\right)\right)^2\in[0,1)$,所以$Var\left(X\right)=E(X-E\left(X\right))^2\in(0,1]$,所以$Cov_s^{var}\left(D^\prime,D\right)=1-Var\left(\frac{n_j^v}{n^v}\right)\in(0,1]$
        \item \textbf{极值敏感性:} 相比于极差的度量方式而言,以方差作为度量方式时,极值对结构覆盖度的影响更温和一些。
    \end{enumerate}

    \subsection{冗余性度量}

    Ma等人在提出的结构覆盖度指标中使用了相似性(similarity)这一概念来衡量集合中任意两个元素之间的可替代性(即在语义层面上互相间的代表性)。如$\sum_{d\in D_j}{\Sim(d_j^\prime,d)}$中的求和项,即$\Sim(d_j^\prime,d)$,其被用来衡量$D_j^\prime$中的元素$d_j^\prime$对于原大数据集合的划分中的$D_j$中的元素在语义层面的代表性,而该求和本身被定义为集合$D_j$的势,而该集合本身也是通过使用相似性这一概念对原大数据集合$D$进行划分。因此,对于原文中所提到的对于多样性的结构度量,对于相似性,即$\Sim(x,y)$函数的选取至关重要,不同的相似性函数会影响到多样性的结构测度的值。

    虽然原文中并没有明确给出对于何种数据使用何样的相似性进行计算,但对于常见的如余弦相似性等相似度其本身往往具有轮换对称性(即关于$y=x$对称),即满足对于$\forall x,y\in D, \Sim\left(x,y\right)=\Sim(y,x)$。该性质在一般情况下自相似性语义角度来考虑是合理的,但是若是自语义的信息量角度来考虑便可以提出不对称的相似性定义,即本文所提出的冗余性(redundancy)。考虑对于评论数据集$D$场景下,存在两条评论$x,y$,其分别表达了不同消费者对于某件商品相似态度的语义,即二者表达了类似的语义,如同时对某件商品给出了正向的评论。若是评论$y$中的语义包含于评论$x$,即对于评论$x$,评论$y$并没有带来信息增益。相反,由于评论$x$的语义信息大于评论$y$,即评论$x$对于评论$y$带来了信息增益。

    \begin{align}
        G\left(\left\{x,y\right\}\right)&-G\left(\left\{x\right\}\right)=0 \\
        G\left(\left\{x,y\right\}\right)&-G\left(\{y\}\right)>0 \\
        G\left(\left\{x\right\}\right)&>G(\{y\})
    \end{align}

    我们使用了语义信息容量函数$G_S$来表达这一概念,其定义在集合$S$上,某一集合$X$在语义上能够反映大数据集合$S$中多少程度上的语义,其值域为$[0,1]$。显然有$G_S\left(S\right)=1$与$G_S\left(\varnothing\right)=0$。\cite{holst2021redundancy}中给出了两类冗余性的定义,我们在此借鉴其对于第一类冗余性(Redundancy Type I)的定义给出了对于大小数据问题下的冗余性的定义。

    \begin{definition}
        冗余性(redundancy):对于$\forall x,y\in S$其冗余性函数为

        \begin{equation}
            r_S\left(x,y\right)=1-[G_S(\left\{x,y\right\}-G_S\left(\left\{x\right\}\right)]
        \end{equation}

        其表示了对于定义在集合$S$上的元素$x,y$,元素$y$相对于元素$x$的语义冗余。
    \end{definition}

    由定义不难知道,对于冗余性函数$r_S(x,y)$,其并不具备如相似性函数一般的轮换对称性,即不一定有$r_D\left(x,y\right)=r_D(y,x)$成立(在某些情况下该式仍然成立)。并且由于语义信息容量函数$G_S$其值域为$[0,1]$,且有$G_S\left(\left\{x,y\right\}\right)-G_S\left(\left\{y\right\}\right)\geq0$成立,即集合$\left\{x,y\right\}\in S$所包含的集合$S$中的语义信息一定不少于集合$\left\{y\right\}\in S$中所包含的集合$S$中的语义信息。

    \begin{proposition}
        冗余函数$r_S\left(x,y\right)=1-[G_S(\left\{x,y\right\})-G_S\left(\left\{x\right\}\right)]$值域为$\left[0,1\right]$。
    \end{proposition}

    \begin{proof}
        由$G_S\left(\left\{x,y\right\}\right)-G_S\left(\left\{y\right\}\right)\geq0$可知,$r_S\left(x,y\right)=1-\left[G_S\left(\left\{x,y\right\}\right)-G_S\left(\left\{x\right\}\right)\right]\le1$。另外,由语义信息容量增益的概念可知,$G_S\left(\left\{x,y\right\}\right)-G_S\left(\left\{x\right\}\right)\le G\left(\left\{y\right\}\right)$,即元素$y$最多为元素$x$带来$G({y})$的语义信息增益。因此,$r_S\left(x,y\right)=1-\left[G_S\left(\left\{x,y\right\}\right)-G_S\left(\left\{x\right\}\right)\right]\geq1-G_S\left(\left\{y\right\}\right)\geq0$。
    \end{proof}

    \begin{proposition}
        冗余函数$r_S\left(x,y\right)=1-[G_S(\left\{x,y\right\})-G_S\left(\left\{x\right\}\right)]$不具备轮换对称性。
    \end{proposition}

    \begin{proof}
        $r_S\left(x,y\right)-r_S\left(y,x\right)=G\left(\left\{x\right\}\right)-G({y})$,当且仅当$G\left(\left\{x\right\}\right)=G({y})$,二者相等。
    \end{proof}

    \begin{proposition}
        冗余函数$r_S\left(x,y\right)=0$当且仅当$G_S\left({y}\right)=1$。
    \end{proposition}

    \begin{proof}
        由$0=1-G_S({y})\le r_S\left(x,y\right)\le1$可知成立。对于$G_S\left({y}\right)<1$,$G_S\left(\left\{x,y\right\}\right)-G_S\left(\left\{x\right\}\right)\le G\left(\left\{y\right\}\right)<1$,可知$r_S\left(x,y\right)\neq0$。
    \end{proof}

    \begin{proposition}
        冗余函数$r_S\left(x,y\right)=1$当且仅当$G\left(\left\{x,y\right\}\right)-G\left(\left\{x\right\}\right)=0$。
    \end{proposition}

    \begin{proof}
        由$r_S\left(x,y\right)$定义可知,$r_S\left(x,y\right)=1-\left[G_S\left(\left\{x,y\right\}\right)-G_S\left(\left\{x\right\}\right)\right]=1$。
    \end{proof}

    在此借助冗余度的概念,类似于原文中集合$D_j$的势给出基于冗余度的集合的势的概念,$n_j^\nu=\sum_{d\in D_j}{r_{D_j}(d_j^\prime,d)}$,其中$d_j^\prime$为集合$D_j$的类别标签。类似地给出$n^v=\sum_{j} n_j^v$,多样性的结构度量$\Cov_s(D^\prime,D)$同原文\eqnref{eq:1}的计算方式。此处,我们给出一种本文中的语义信息容量函数及冗余度函数的实例。

    \begin{align}
        G_S({x})&=\frac{1}{|S|}\sum_{d\in S}{\Sim(x,d)} \\
        G_S({x,y})&=\frac{1}{|S|}\sum_{d\in S}\max\left\{\Sim(x,d),\Sim(y,d)\right\} \\
        r_S\left(x,y\right)&=1-\frac{1}{\left|S\right|}\sum_{d\in S}\left\{\max\left\{\Sim\left(x,d\right),\Sim\left(y,d\right)\right\}-\Sim\left(x,d\right)\right\} \label{eq:4}
    \end{align}

    \section{计算示例}

    \begin{example}
        容器中包含红球、黄球、蓝球、黑球,数量分别为$100$、$200$、$300$、$400$个,现有如下四种抽样结果:

        \begin{enumerate}
            \item $10$个黑球
            \item $4$个红球、$3$个黄球、$2$个蓝球、$1$个黑球
            \item $1$个红球、$2$个蓝球、$3$个黄球、$4$个黑球
            \item $1$个红球、$1$个蓝球、$1$个黄球、$1$个黑球
        \end{enumerate}

        分别使用$\Cov_s^\text{range}(D', D), \Cov_s^\text{var}(D', D)$作为衡量抽样$D'$在全集$D$上的结构覆盖度。计算结果如表所示

        \begin{table}[ht]
            \centering
            \caption{结构覆盖度计算结果}
            \begin{tabular}{ccccc}
                \toprule
                度量指标 & 1 & 2 & 3 & 4 \\
                \midrule
                $\Cov_s^\text{range}$ & 0 & \textbf{0.625} & 1 & \textbf{0.7} \\
                $\Cov_s^\text{var}$ & 0 & \textbf{0.9866} & 1 & \textbf{0.9833} \\
                $\Cov_s$ & 0 & 0.8 & 1 & 0.923 \\
                \bottomrule
            \end{tabular}
        \end{table}
    \end{example}

    可以发现两种度量方法都能在结构覆盖度角度上选出抽样3最能代表大数据集。但是在抽样2和抽样4的结构覆盖度的排序上出现了差异。在$\Cov_s^{\text{range}}(D', D)$中,只关注结构覆盖差异最大的部分结构的势来度量结构覆盖度,只关注类别结构的势的极值,抽样4相比于抽样2,最小的类别结构的势变大了,更接近最大类别结构的势,所以认为抽样4在结构覆盖度上优于抽样2。而$\Cov_s^{\text{var}}\left(D^\prime,D\right)$因为衡量了每个结构的势来计算结构覆盖度,虽然抽样4相比于抽样2,最小的类别结构的势变大,但是其他类别结构的势之间的离散程度变大了,所以总体上使得不同类别结构的势的分布更离散了,所以结构覆盖度减小了。

    \begin{example}
        已知一组文档$d_i \in\{\textit{a,b,c,d,e,f}\}$,其相似度矩阵为$A = (\Sim(d_i, d_j))_{6\times 6}$:

        \begin{equation}
            A = \begin{bmatrix}
                1 & 0.95 & 0.03 & 0.05 & 0.12 & 0.21 \\
                0.95 & 1 & 0.13 & 0.08 & 0.15 & 0.01 \\
                0.03 & 0.13 & 1 & 0.87 & 0.92 & 0.78 \\
                0.05 & 0.08 & 0.87 & 1 & 0.85 & 0.95 \\
                0.12 & 0.15 & 0.92 & 0.85 & 1 & 0.77 \\
                0.21 & 0.01 & 0.78 & 0.95 & 0.77 & 1 \\
            \end{bmatrix}
        \end{equation}

        计算$D$中各元素相互的的冗余性指标和抽样$D' = \{e, a\}$的结构性度量。
    \end{example}

    根据式\eqnref{eq:4}有

    \begin{equation}
        a_{ij}' = 1-\frac{1}{6}\sum_{k}\left(\max\left\{a_{ik},a_{jk}\right\}-a_{ik}\right)
    \end{equation}

    计算得到

    \begin{equation}
        A' = \left(a'_{ij}\right)_{6\times 6} = \begin{bmatrix}
            1 & 0.965 & 0.473 & 0.457 & 0.478 & 0.485 \\
            0.958 & 1 & 0.467 & 0.450 & 0.472 & 0.478 \\
            0.702 & 0.702 & 1 & 0.947 & 0.968 & 0.920 \\
            0.697 & 0.697 & 0.958 & 1 & 0.943 & 0.965 \\
            0.729 & 0.720 & 0.982 & 0.945 & 1 & 0.930 \\
            0.712 & 0.712 & 0.918 & 0.952 & 0.915 & 1 \\
        \end{bmatrix}
    \end{equation}

    分别使用$\Cov_s, \Cov_s^{\text{range}}, \Cov_s^{\text{var}}, \Cov_s^r$计算得到表\ref{tbl:2}

    \begin{table}[ht]
        \centering
        \caption{不同指标下的结构相似度}
        \begin{tabular}{cc}
            \toprule
            结构性度量指标 & 结构相似度 \\
            \midrule
            $\Cov_s$ & 0.99971 \\
            $\Cov_s^{\text{range}}$ & 0.97996 \\
            $\Cov_s^{\text{var}}$ & 0.9998 \\
            $\Cov_s^r$ & 0.9224 \\
            \bottomrule
        \end{tabular}
        \label{tbl:2}
    \end{table}

    \section{总结}

    综合来看,本文在Ma等人所提出的多样性结构度量的基础上进行了分析与改进,主要的工作内容有以下几点:

    \begin{enumerate}
        \item 本文分析了Ma等人提出的基于信息熵的多样性结构度量指标的合理性及其如何反应整体的信息结构;
        \item 本文在Ma等人的基础上,提出了基于方差与极差指标的结构性度量并论证了一些基本性质,最后给出了简单的例子计算二者与原基于信息熵结构度量指标的异同;
        \item 本文受Holst等人的启发,提出了基于第一类冗余定义的冗余度及其改进的势和多样性的结构度量,同样在简单的例子上进行了计算,并比较其与原文中结构度量指标的异同;
    \end{enumerate}

    \nocite{yager1992specificity}
    \nocite{yager1998measures}
    \nocite{yager2008measures}
    \nocite{higashi1982measures}
    \nocite{higashi1983notion}

    \bibliographystyle{plain}
    \bibliography{HW03}
\end{document}