\documentclass{../notes}

\title{高级信息系统 HW02}

\newcommand{\irisresult}{
    (1, 681.37)
    (2, 152.35)
    (3, 78.85)
    (4, 57.26)
    (5, 46.45)
    (6, 39.04)
    (7, 34.53)
    (8, 30.32)
    (9, 27.86)
    (10, 26.49)
}

\begin{document}
    \maketitle

    \paragraph*{1}

    当$n=1$时,仅考虑$m$。对于数据项集合,原问题可以转化为从集合$I$生成二元组$\langle X, Y\rangle$满足$X\subset I, Y\subset I, X\cap Y=\varnothing, X\cup Y\not =\varnothing$。由题,$I$的基数为$m$。则对于每个元素$i\in I$,有三种情况,即(1)$i\in X$,(2)$i\in Y$,(3)$i\not \in X, i\not \in Y$,因此共有$3^m$种情形。考虑约束条件$X\not = \varnothing, Y\not = \varnothing$,去掉$2\times 2^{m} - 1$种情况。由数据项集合生成所有可能的$X, Y$集合的复杂度为$O(3^{m} - 2\times 2^{m} + 1) = O(3^m)$

    对于数据记录集合,每条数据记录中最多包含$m$个数据项,共有$n$条记录,因此由$n$条记录生成关联规则的计算复杂度为$O(n)O(3^m) = O(n\cdot 3^m)$

    \separate

    \paragraph*{2} 

    $r'$在$R$中是冗余的,证明如下:

    对于Apriori算法,设支持度阈值$\alpha$,置信度阈值$\beta$;数据项“啤酒”=$A$、“尿布”=$B$、“硬盘”=$C$;已知$A\Rightarrow BC$为有效的关联规则要证$A\Rightarrow B$、$A\Rightarrow C$均为有效的关联规则。
    
    已知$\prob A\geq \alpha, \prob B\geq \alpha, \prob C\geq \alpha$且$\cprob{BC}A = \prob{ABC}/\prob A\geq \beta$,原问题即证明$\cprob BA = \prob{AB}/\prob A\geq \beta, \cprob CA = \prob{AC}/\prob C\geq \beta$。
    
    设$X, Y$为任意事件,则显然有$P(X|Y)\leq 1$,即$P(XY)\leq P(Y)$。
    
    由于$P(AB)\geq P(ABC), P(AC)\geq P(ABC)$,即

    \begin{equation}
        \begin{aligned}
            \cprob BA &= \prob{AB}/\prob A\geq \prob{ABC}/\prob A \geq \beta \\
            \cprob CA &= \prob{AC}/\prob A\geq \prob{ABC}/\prob A \geq\beta
        \end{aligned}
    \end{equation}
    
    因此$A\Rightarrow B, A\Rightarrow C$。


    \separate

    \paragraph*{3 (关于算法参数和特点)}

    \subparagraph*{1.} 对于K-Means算法,设向量空间$\R^m$内的样本点集$X_i = (x_1, x_2, \dots x_m), 1\leq i \leq n$。使用样本点之间的距离$\Vert X_i - X_j\Vert_2$表示$X_i, X_j$之间的距离,定义损失函数$f$如下:

    \begin{equation}
        f(K) := \sum_{k=1}^K \sum_{i=1}^{r_k} \left\Vert X_{s_{ki}} - \bar X_{k} \right\Vert _2^2
    \end{equation}

    式中$r_k$表示第$k$个聚类中的元素数量,$X_{s_{ki}}$表示第$k$个聚类中的第$i$个点,$\bar X_k$表示迭代算法达到稳定时第$k$个聚类中心点,损失函数$f$反映了聚类中各点到达聚类中心点距离的平方和。当改变$K$时,损失函数也会随之改变。

    设实际聚类数量$K$,真实聚类数量$K_0$,当$K<K_0$时,聚类算法将实际归属多个聚类的样本点划分到一个聚类中,而由于不同聚类内的样本点通常有较大的差异,使得损失函数$f$取值较大。而当$K$从$K_0 - 1$增加到$K_0$时,每个样本点能够划分到属于自己的聚类内,聚类内的样本点差异较小,因此损失函数会出现明显下降。当$K\geq K_0$时,由$K$个聚类进一步划分到$K + 1$个聚类时,实际上是将已经存在的聚类一分为二,因此损失函数会有下降,但不会出现较大幅度的下降。
    
    图\ref{fig:k-to-f}列出了在Iris数据集上K-Means算法对不同聚类数量输出的$f$值。由图可以看出,当聚类数量由1上升到2、由2上升到3时,损失函数均出现较为明显的下降,而当聚类数量从3上升到4时,损失函数的变化则较不明显,因此Iris数据集的聚类数量应当为3。

    \begin{figure}
        \centering
        \begin{tikzpicture}
            \begin{axis}[sharp plot]
                \addplot coordinates {\irisresult};
            \end{axis}
        \end{tikzpicture}
        \caption{在Iris数据集上不同$F$对应的$f$值}
        \label{fig:k-to-f}
    \end{figure}

    \subparagraph*{2.}

    当数据类簇不规则时,类簇内数据点的分布应当满足如下条件:

    \begin{enumerate}
        \item 数据点之间的距离应当足够紧密,从而数据点可以与周围其他簇或噪声点产生明显区别。即对于簇$X$,$\forall x\in X$,应有$\forall y\not \in X, \Vert x, y\Vert_2$较小。
        \item 数据点分布不均匀,即对于簇$X$,进行随机抽样得到子集$X_1, X_2\subseteq X$,$X_1, X_2$样本均值的差异$\Vert \mu_{1} - \mu_{2}\Vert_2$较大。
    \end{enumerate}

    对于DBSCAN算法在不规则数据集上的识别问题,可以将该问题拆分为:

    \begin{enumerate}
        \item \textbf{DBSCAN可以在不规则数据集上得到结果吗?} 与K-Means算法假设数据点在聚类内均匀分布不同,DBSCAN不对数据的分布作任何假设。也即,DBSCAN不依赖样本的统计指标。因此DBSCAN可以将不规则的点集识别为簇。
        \item \textbf{DBSCAN得到的结果是有效的簇吗?} DBSCAN通过超参数,按照样本点之间的距离对样本点进行划分,因此识别的是分布密度足够高(i.e. 样本点之间距离/差异足够小)的样本点集合。从而DBSCAN算法可以得到有效的聚类簇。
    \end{enumerate}
    
    \paragraph*{4} 设置$\beta > 50\%$时则不会发生冲突,证明如下:

    已知频繁项$X, C_1\in S$满足$\prob X\geq \alpha, \prob{C_1}\geq \alpha, \prob{XC_1}\geq \alpha$,且$\cprob{C_1}X\geq \beta, \beta > 0.5$,证明$\not \exists C_2\in S, C_1\cap C_2 = \varnothing, \cprob{C_2}X\geq \beta$

    假设$\exists C_2\in S, C_1\cap C_2 = \varnothing, \cprob{C_2}X\geq \beta$,由于$C_1\cap C_2 = \varnothing$,则$(XC_1)\cap (XC_2) = \varnothing,$有

    \begin{equation}
        \begin{aligned}
            \prob{X(C_1\cup C_2)} &= \prob{XC_1} + \prob{XC_1} \\
            &= \cprob{C_1}X\prob X + \cprob{C_2}X\prob X \\
            &> 0.5\prob X + 0.5\prob X \\
            &> \prob X
        \end{aligned}
    \end{equation}

    推出$\cprob{C_1\cup C_2}X = \prob{X(C_1\cup C_2)} / \prob X > 1$,矛盾。

    反之,当$\beta$足够小时,可能存在$C_1, C_2$使得$\cprob{C_1\cup C_2}X = \prob{X(C_1\cup C_2)} / \prob X$,从而同时推出$X\Rightarrow C_1, X\Rightarrow C_2$。
\end{document}